\section{Discussion}
In this section we discuss the implications of our research (Section 9.1) and the limitations of our work (Section 9.2).

\subsection{Implications}
Our work has implications for several research threads.

\textbf{Power Law of Participation.} Ross Mayfield coined the term \lq Power Law of Participation\rq\---when a small number of community members participate in high engagement activities, while the larger community participate in low threshold activities. We observe power-law of participation in Stack Exchange markets. Stack Exchange supports varieties of activities ranging from low-threshold (e.g., voting) to high-engagement (e.g., collaborative editing, and linking similar questions); with a small fraction of users participating in high-engagement activities. 

We assert that both low-threshold and high-engagement activities are required for a knowledge market's survival, and should proportionately increase with the increase in number of participants. However, in reality, for most knowledge markets, the user community contributing high-engagement activities does not scale with the system size---which creates a gap between market supply and demand for high-engagement activities, and consequently affects market health. 

%He observed that for communities such as Wikipedia, Flickr, and Digg, several high engagement activities (e.g., Wikipedia editing) require collaboration among users, whereas low engagement activities are somewhat unattached.

%\textbf{Controllability.} In a complex network, controllability refers to the identification of the driver nodes that can guide the system's entire dynamics. Controllability is a function of degree distribution 

\textbf{Microfoundations.}
The size-dependent distribution of user contribution implies that users who join a community later in its lifecycle exhibit different behavior than those who were present from the beginning. This very well may imply that the distribution of individual user behaviors (not just their overall production) is \emph{also} a function of the system size. We should expect to see more a stable user behavior distribution over time for networks that appear to be more scale-insensitive; preliminary results suggest that this may indeed be the case.

\subsection{Limitations}
Now, we discuss several limitations of our work. First, the economic production models do not account for user growth. While there exists several user growth models for two-sided markets~\cite{Kumar2010}, membership based websites~\cite{Ribeiro2014}, and online social networks~\cite{zang2016}, it would be useful to introduce an economic user growth model that properly complements the production models. A potential direction in this research is to apply Malthusian growth model. Second, the production models inherit the fundamental assumptions of macroeconomics such as an aggregate is homogeneous (without looking into its internal composition), and aggregates are functionally related etc. It would be useful to empirically study these assumptions for knowledge markets.


%\subsection{Complementary Models} 
%We considered several alternative models to comprehend the content generation dynamics in knowledge markets. Our first attempt was to model user content generation as \lq self-exciting\rq\ point process. In this attempt, we designed and implemented several variants of Hawkes process. Our second attempt was to model user content generation using \lq stage-structured\rq\ projection matrix. In this attempt, we build variants of Leftkovitch matrix. While both models showed promise, they failed to meet one or more requirements reported in Section 3.

\iffalse
\textbf{Hawkes Process.} The Hawkes process is a mathematical model for self-exciting processes that models a sequence of arrivals of some event over time, e.g., natural disasters, gang violence, trade orders.  Each arrival excites the process in the sense that the chance of a subsequent arrival is increased for some time period after the initial arrival. 

Content generation in Stack Exchange websites can be conceptualized as Hawkes process with multiple event types. Here, arrival of certain type of event (e.g., questions) excites the arrival of other types of event (e.g., answers, comments). 

A univariate Hawkes process is defined to be a self-exciting temporal point process $N$ whose conditional intensity function $\lambda = \lambda(t)$ is defined to be

 $$\lambda(t) = \mu(t)+\sum_{i:\tau_i<t}g(t-\tau_i),$$
 
where $\mu(t)$ is the background rate of the process  $N$, where $\tau_i$ are the points/events in time occurring prior to time t, and where $g$ is a function which governs the clustering density of N. 

The multi-dimensional Hawkes process is defined by a $U$-dimensional point process $N_t^u, u =1, . . . , U$, with the conditional intensity for the $u$-th dimension expressed as follows:
$$\lambda_u(t) = \mu_u + \sum_{i:\tau_i<t} g_{uu_i}(t-\tau_i),$$

where $\mu_u \ge 0$ is the base intensity for the $u$-th Hawkes process. The kernel $g_{uu^\prime}(t) \ge 0$ captures the mutually exciting property between the $u$-th and $u^\prime$-th dimension. Intuitively, it captures the dynamics of influence of events occurred in the $u^\prime$-th dimension to the $u$-th dimension. Larger value of $g_{uu^\prime}(t)$  indicates that events in $u^\prime$-th dimension are more likely to trigger a event in the $u$-th dimension after a time interval $t$.

\textbf{Leftkovitch Matrix.} \textcolor{red}{A brief description of the Leftkovitch Matrix based models.}
\fi


