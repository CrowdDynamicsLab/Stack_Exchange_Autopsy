\documentclass[sigconf,anonymous]{acmart}

\usepackage{booktabs} % For formal tables
% My packages
\usepackage{tikz}
\usetikzlibrary{bayesnet}
\usepackage[normalem]{ulem}
\usepackage{wrapfig}
\usepackage{xcolor}
\usepackage{array}
\usepackage{multirow}
\usepackage[inline]{enumitem}
\newcommand{\hs}[1]{\textcolor{red}{Hari: #1}}

% Copyright
%\setcopyright{none}
%\setcopyright{acmcopyright}
%\setcopyright{acmlicensed}
%\setcopyright{rightsretained}
%\setcopyright{usgov}
%\setcopyright{usgovmixed}
%\setcopyright{cagov}
%\setcopyright{cagovmixed}


% DOI
%\acmDOI{10.475/123_4}

% ISBN
%\acmISBN{123-4567-24-567/08/06}

%Conference
\acmConference[WWW'18]{The Web Conference 2018}{April 2018}{Lyon, France}
\acmYear{2018}

%\copyrightyear{2016}

%\acmPrice{15.00}


\begin{document}
\title[Too Big to Succeed]{Too Big to Succeed: Understanding Successes and Failures at Scale in Knowledge Markets}

\iffalse
\titlenote{Produces the permission block, and
  copyright information}
\subtitle{Extended Abstract}
\subtitlenote{The full version of the author's guide is available as
  \texttt{acmart.pdf} document}
\fi

\author{Anonymous Author}
\affiliation{%
  \institution{Anonymous Institution}
  \city{City, Country}
}
\email{e-mail}

\iffalse
\author{Ben Trovato}
\authornote{Dr.~Trovato insisted his name be first.}
\orcid{1234-5678-9012}
\affiliation{%
  \institution{Institute for Clarity in Documentation}
  \streetaddress{P.O. Box 1212}
  \city{Dublin}
  \state{Ohio}
  \postcode{43017-6221}
}
\email{trovato@corporation.com}

\author{G.K.M. Tobin}
\authornote{The secretary disavows any knowledge of this author's actions.}
\affiliation{%
  \institution{Institute for Clarity in Documentation}
  \streetaddress{P.O. Box 1212}
  \city{Dublin}
  \state{Ohio}
  \postcode{43017-6221}
}
\email{webmaster@marysville-ohio.com}

\author{Lars Th{\o}rv{\"a}ld}
\authornote{This author is the
  one who did all the really hard work.}
\affiliation{%
  \institution{The Th{\o}rv{\"a}ld Group}
  \streetaddress{1 Th{\o}rv{\"a}ld Circle}
  \city{Hekla}
  \country{Iceland}}
\email{larst@affiliation.org}

\author{Lawrence P. Leipuner}
\affiliation{
  \institution{Brookhaven Laboratories}
  \streetaddress{P.O. Box 5000}}
\email{lleipuner@researchlabs.org}

\author{Sean Fogarty}
\affiliation{%
  \institution{NASA Ames Research Center}
  \city{Moffett Field}
  \state{California}
  \postcode{94035}}
\email{fogartys@amesres.org}

\author{Charles Palmer}
\affiliation{%
  \institution{Palmer Research Laboratories}
  \streetaddress{8600 Datapoint Drive}
  \city{San Antonio}
  \state{Texas}
  \postcode{78229}}
\email{cpalmer@prl.com}

\author{John Smith}
\affiliation{\institution{The Th{\o}rv{\"a}ld Group}}
\email{jsmith@affiliation.org}

\author{Julius P.~Kumquat}
\affiliation{\institution{The Kumquat Consortium}}
\email{jpkumquat@consortium.net}
\fi

% The default list of authors is too long for headers}
%\renewcommand{\shortauthors}{B. Trovato et al.}

\begin{abstract}
 In this paper, we analyze a large group of community question answering
 (CQA) websites on Stack Exchange platform through an economic lens by
 modeling them as knowledge markets. While many of these websites
 continually grow over time, we demonstrate that they can, and often will,
 fail at scale. Metrics of CQA health such as the ratio of answers to
 questions, the percentage of answered questions, and the percentage of
 questions with an accepted answer frequently decline as a function of
 system size (no. of participants) on these platforms. 
 Understanding why this phenomenon
 occurs is a necessary step towards preventing the failures at scale. Our main contribution is to model CQA websites as knowledge markets, and to provide insight on the relationship between size and health of these markets. To
 this end, we explore a set of interpretable economic production models to
 capture content generation dynamics in knowledge markets by fitting each
 to the Stack Exchange data. The best performing of these,
 well-known in economic literature as Cobb-Douglas equation,
 provides an intuitive explanation for content generation in the knowledge
 markets. Specifically, it shows that \begin{enumerate*}
  \item factors of content generation such as user participation and content dependency have
   \emph{constant elasticity}---a percentage increase in any of the inputs
   leads to a constant percentage increase in the output,
  \item in many markets, factors exhibit \emph{diminishing returns}---the
   incremental, marginal output decreases as the input is incrementally
   increased,
  \item markets vary according to their \emph{returns to scale}---the
   increase in output resulting from a proportionate increase in all
   inputs, and finally
  \item many markets exhibit \emph{diseconomies of scale}---measures of
   health decrease as a function of overall
   system size
 \end{enumerate*}.
\end{abstract}

\iffalse
%
% The code below should be generated by the tool at
% http://dl.acm.org/ccs.cfm
% Please copy and paste the code instead of the example below.
%
\begin{CCSXML}
<ccs2012>
 <concept>
  <concept_id>10010520.10010553.10010562</concept_id>
  <concept_desc>Computer systems organization~Embedded systems</concept_desc>
  <concept_significance>500</concept_significance>
 </concept>
 <concept>
  <concept_id>10010520.10010575.10010755</concept_id>
  <concept_desc>Computer systems organization~Redundancy</concept_desc>
  <concept_significance>300</concept_significance>
 </concept>
 <concept>
  <concept_id>10010520.10010553.10010554</concept_id>
  <concept_desc>Computer systems organization~Robotics</concept_desc>
  <concept_significance>100</concept_significance>
 </concept>
 <concept>
  <concept_id>10003033.10003083.10003095</concept_id>
  <concept_desc>Networks~Network reliability</concept_desc>
  <concept_significance>100</concept_significance>
 </concept>
</ccs2012>
\end{CCSXML}

\ccsdesc[500]{Computer systems organization~Embedded systems}
\ccsdesc[300]{Computer systems organization~Redundancy}
\ccsdesc{Computer systems organization~Robotics}
\ccsdesc[100]{Networks~Network reliability}


\keywords{ACM proceedings, \LaTeX, text tagging}
\fi


\maketitle
%\newpage
\section{Introduction}
\textcolor{blue}{To be written.}
Content generation and subsequent consumption is a key feature of the social web experience. Every second, on average, several thousand tweets are tweeted on Twitter, which corresponds to several hundred million tweets per day and several hundred billion tweets per year. The content generation stats are in similar scale for other massive social media platforms such as Facebook and YouTube. We continue to witness the Web 2.0 era in the light of these user-generated content (UGC) platforms, where content generation is instrumental in preserving sustainability of the platforms-- serving \lq nowness\rq\ in the social web experience. 

While content generation is at the heart of sustainable UGC platforms, our understanding of what factors are effective at encouraging content generation and associated growth is yet limited. To date, growth of UGC platforms has been studied mainly from the perspective of population dynamics, with researchers emphasizing interacting population based models to explain user growth in these platforms. These models, however, only consider the interaction among users, and ignores the interaction among users and content-- consequently, can not explain content growth. There is a growing body of empirical studies that analyze patterns of user content generation-- primarily temporal patterns-- revealing useful insights. However, even with these insights, there is a gap in realizing the factors of content growth and corresponding dependency. What is missing from the picture are models of content growth that clearly establish the relationship between content growth and associated factors. Developing such models is important for understanding content growth dynamics, predicting future growth, and identifying platforms that will sustain for a long period of time.
\section{Related Work}
Our work draws from, and improves upon, several research threads.

\textbf{Sustainability.} There is a relatively recent body of work studying sustainability in knowledge markets. Notably, \citet{srba2016stack} conducted a case study on why StackOverflow, the largest and oldest of the networks on StackExchange, is failing. They reveal some insights into market failure such as novice and negligent users generating low quality content perpetuating the decline of the network. However, they do not present a systematic way to understand and prevent failures in knowledge markets. \citet{wu2016} introduced a framework for understanding the user strategies in a knowledge market---revealing the importance of diverse user strategies for sustainable markets. %In this paper, we present an alternative model that provides many interesting insights including sustainability.

\textbf{Activity Dynamics.} There have been a number of papers on modeling the activity dynamics of online platforms~\cite{wu2011, anderson2012, walk2016}. \citet{walk2016} modeled user-level (micro-scale) activity dynamics in StackExchange using two factors: intrinsic activity decay, and positive peer influence. However, the model proposed there can not be used to study content-driven success or failure of online platforms because \begin{enumerate*}
    \item it does not reveal insights on the collective platform dynamics, and
\item it does not concentrate on the eventual success or failure of a platform\end{enumerate*}. \citet{wu2011} proposed a discrete generalized beta distribution (DGBD) model that reveals several insights into the collective dynamics, notably the concept of a size dependent distribution. In this paper, we improve upon the concept of a size dependent distribution.

\textbf{Scale Study.} \citet{lin2017} examined Reddit communities to characterize the effect of user growth in voting patterns, linguistic patterns, and community network patterns. The study reveals that these patterns do not change much after a massive growth in the size of the user community. In this paper, we examine the consequence of scale on knowledge markets from a different perspective by using a set of health and stability metrics.

\textbf{Stability.} Successes and failures of networks have been studied from the perspective of stability. \citet{patil2013} studied the dynamics of group stability in social networks. They define stability based on the average increase or decrease in member growth. Our paper examines stability in a different manner---namely, by considering the relative exchangability of users as a function of scale.

\textbf{User Growth.} Successes and failures of networks have also been widely studied from the perspective of user growth~\cite{Kumar2006, Backstrom2006, kairam2012, zang2016}. Notably, \citet{Backstrom2006} studied the mechanisms that underpin of how users join communities in a social network; \citet{kairam2012} examined diffusion (growth via social ties) and non-diffusion (growth without social ties) processes to design models that predict the longevity of social groups. These works, however, do not model active user growth. To capture active user growth, \citet{Ribeiro2014} proposed a daily active user (DAU) prediction model for membership based websites; the model classifies membership based websites as sustainable and unsustainable. While this perspective is important, we argue that the success and failure of networks based on their \emph{content production} can perhaps be more meaningful.

\textbf{Content Generation.} Analyzing patterns of user content generation is crucial for developing principled content generation models. \citet{Guo2009} analyzed three social networks and revealed the stretched exponential distribution of user contribution. In this paper, we argue that the distribution of user contribution is size-dependent in CQA networks.

%\subsection{Modeling User Activity and Content}
%First, we discuss the prior works in modeling user activity and content generation in online platforms.

%\subsection{Modeling Network Success and Failure}
%Now, we discuss the related works on network success and failure.

\textbf{Modeling CQA Websites.} \citet{furtado2013} explore user behavior
profiles and their dynamics in five StackExchange networks by performing an
agglomerative clustering on manually specified user attributes. This can be
viewed as a providing an understanding of behavior dynamics at a ``micro''
level (at the level of individual users). A major difference of our work
from theirs is that in this paper we take a ``macro'' view of the behavior
dynamics of CQA networks by looking at the behavior of an entire network as
a function of its user population. \citet{Kumar2010} proposes taking an
economic view of CQA networks, like we do, but under their formulation
users must be on one-and-only-one side of the two-sided market. Our model
differs in that it does not make any assumption about the presence or
absence of overlap in the group of users that provide answers and the group
of users that provide questions. Furthermore, their model does not provide
a systematic way of understanding the amount of \emph{content} the market
generates. Rather, they focus on the growth of the two kinds of users.
\citet{Yang2015} identifies the scalability problem of CQA networks that we
study here---namely, the volume of question content eventually subsumes the
capacity of the answerers within the community. Understanding and modeling
this phenomenon is one of the goals of this paper.

%\textbf{Modeling Markets.} 


\section{Problem Formulation} 
In this section we describe the requirements for designing a model to understand the successes and failures of knowledge markets. Since content generation and consumption is the key means to measure a knowledge markets' success or failure, we aim to build a model to better understand the content generation dynamics. Specifically, we opt for a model with the following desired properties.

%To better understand the content generation dynamics and subsequent success/failure of a knowledge market, we opt for a model with the following desired properties.

\textbf{Macro-scale.} The model should capture content generation dynamics at aggregate level. 

\textbf{Explanatory.} The model should give us some deeper understanding of how an aggregate knowledge market behaves.

\textbf{Predictive.} The model should allow us to make predictions about future content generation and resultant success or failure.

\textbf{Minimalistic.} The model should have as few parameters as necessary, and still closely reflect the observed reality.

\textbf{Comprehensive.} The model should encompass content generation dynamics for different content types (e.g., question, answer, comment) in varieties of knowledge markets.

In remaining sections we propose models that meet the aforementioned requirements, and show that our best-fit model accurately reflects the content generation dynamics and resultant successes and failures of many real-world knowledge markets.

\section{Modeling Knowledge Markets}
In this section we introduce economic production models to capture content generation dynamics in real-world knowledge markets. We first draw an analogy between economic production and content generation (Section 4.1), and then report the content generation factors in knowledge markets (Section 4.2). Next, we concentrate on the knowledge markets in Stack Exchange networks---presenting production models for different content types (Section 4.3).

\subsection{Production Analogy} 
Economic production well describe content generation in knowledge markets. In Economics, \emph{production} is defined as the process by which human labor is applied, usually with the help of tools and other forms of capital, to produce useful goods or services---the \emph{output}. We assert that participants of a knowledge market function as labor to generate contents such as questions and answers. Similar to the economic output, these knowledge contents contribute to the utility received by the market participants. Motivated by this analogy, we use macroeconomic production models to capture content generation dynamics in knowledge markets. In these models, instead of directly modeling content generation as a dynamic process (function of time), we model it in terms of associated factors, which themselves are dynamic.

%Motivated by the macroeconomic production models, we focused on designing factor based model for user content generation in CQA platforms. Instead of directly modeling user content generation as a dynamic process (function of time), we model it in terms of associated factors, which themselves are dynamic. From this point forward, we report the factors of content generation for different content types, discuss basis functions to capture the effect of a single factor and aggregate functions to capture the interaction among multiple factors, and introduce alternative models based on the basis and aggregate functions.

%We conceptualize content generation in knowledge markets as economic production. In Economics, \emph{production} is defined as the process by which human labor is applied, usually with the help of tools and other forms of capital, to produce useful goods or services--the output. An output is a good or service that has value and contributes to the utility received by individuals. We assert that users of a knowledge market function as labor to generate content, which has value and contributes to the utility received by the users. In the following subsections, we report the factors of production or inputs for different types of content generation, present a production function to capture the relationship between the output and the inputs of production, and introduce a set of growth models that capture long-run growth of different types of content.

\subsection{Factors of Content Generation} 
%We recognize the key factors of content generation in knowledge markets. In economics, \emph{factors} refer to inputs that are used in the production process to produce output. 
There are two key factors that affect content generation in knowledge markets, namely user participation and content dependency.

\textbf{User Participation.} The number of active users is the most important factor in deciding the quantity of generated content. The participation of more users induce more questions, answers, and other contents.

\iffalse
\begin{wrapfigure}{R}{0.15\textwidth}
\vspace{-\baselineskip}
\centering
\scalebox{.6}{
  \tikz{ %
  	\node[obs] (answerers) {$U_a$} ; %
    \node[obs, right=of answerers] (answers) {$N_a$} ; %
    \node[obs, right=of answers] (comments) {$N_c$} ; %
    \node[obs, above=of answerers] (askers) {$U_q$} ; %
    \node[obs, right=of askers] (questions) {$N_q$} ; %
    \node[obs, right=of questions] (commenters) {$U_c$} ; %    
    \edge {askers} {questions} ; %
    \edge {questions, answerers} {answers} ; %
    \edge {questions, answers, commenters} {comments} ; %   
  }}
  \caption{SE factors}
  \vspace{-\baselineskip}
  \label{fig:content_factors}
\end{wrapfigure}
\fi

\textbf{Content Dependency.} While user participation is vital for content generation, content dependency also affects the quantity of generated content for different types. Content dependency implies the dependency of a type of content on other type of content(s). For example, answer generation relies on question generation. In absence of questions, there will be no answers, even in the presence of many potential answerers. 

\subsection{Modeling Markets in Stack Exchange}
\textcolor{blue}{This subsection requires restructuring.}
We design production models for different types of contents in Stack Exchange. Each market in Stack Exchange primarily generates three types of contents: question, answer, and comment. Based on the factors in Section 4.2, we propose the following relationships for these contents types (See Table~\ref{tab:notations} for notations).

\begin{table}[thb]
	\vspace{-0.5\baselineskip}
	\caption{Notations used in the model}
    \vspace{-\baselineskip}
    \label{tab:notations}
	\begin{center}    
	\begin{tabular}{cl}
	\toprule Symbol & Definition\\ \midrule
	$U_q(t)$ & \# of users who asked questions at time $t$\\ 
	$U_a(t)$ & \# of users who answered questions at time $t$\\
	$U_c(t)$ & \# of users who made comments at time $t$\\
	$N_q(t)$ & \# of active questions at time $t$\\
	$N_a(t)$ & \# of answers to active questions at time $t$\\
	$N_c^q(t)$ & \# of comments to active questions at time $t$\\
	$N_c^a(t)$ & \# of comments to active answers at time $t$\\
    $N_c(t)$ & \# of comments to active questions/answers at time $t$\\
	$f_w$ & The functional relationship for content $w$\\ \bottomrule
	\end{tabular}
    \end{center}
    \vspace{-\baselineskip}   
\end{table}

%Figure~\ref{fig:content_factors} shows these factors using notations from Table~\ref{tab:notations}. 

\iffalse
\begin{figure}[hbt]
  \centering
  \scalebox{.75}{
  \tikz{ %
  	\node[obs] (answerers) {$U_a$} ; %
    \node[obs, right=of answerers] (answers) {$N_a$} ; %
    \node[obs, right=of answers] (comments) {$N_c$} ; %
    \node[obs, above=of answerers] (askers) {$U_q$} ; %
    \node[obs, right=of askers] (questions) {$N_q$} ; %
    \node[obs, right=of questions] (commenters) {$U_c$} ; %    
    \edge {askers} {questions} ; %
    \edge {questions, answerers} {answers} ; %
    \edge {questions, answers, commenters} {comments} ; %   
  }}
  \caption{Factors of content generation in Stack Exchange}
  \label{fig:content_factors}
\end{figure}
\fi

There is a single factor in generating questions: users who ask questions (aka askers).
\begin{equation*}
N_q = f_q(U_q)
\end{equation*}

There are two key factors in generating answers: questions, and users who answer questions (aka answeres). 
\begin{equation*}
N_a = f_a(N_q, U_a)
\end{equation*}

There are three key factors in generating comments: questions, answers, and users who make comments on these questions and answers (aka commenters). 
\begin{equation*}
N_c^q = f_{c^q}(N_q, U_c)
\end{equation*}
\begin{equation*}
N_c^a = f_{c^a}(N_a, U_c)
\end{equation*}
\begin{equation*}
N_c = N_c^q + N_c^a
\end{equation*}

The aforementioned relationships imply that the number of generated content of each type depends on the function describing its factor dependent growth, and the availability of factor(s). These relationships embody three critical assumptions. First, they assume that different content types interact only through their use of factors. Second, they assume that the functional relationships depend on the consumption/usage of each factor--- how each content type consumes/utilizes each of its factors. Third, they assume that the functional relationships depend on the interaction among the factors--- how the factors of a particular content type interact. Based on these assumptions, we specify the functional relationships by first choosing a basis function to capture how a content type consumes its factor(s), and then choosing an interaction type to capture the interaction among factors.

\textbf{Basis Function.} We use a basis function to capture the effect of a given factor on a particular content type. While there is a variety of basis functions available for regression, we consider three basis functions widely used in economics and growth modeling, namely power-- $g(x) = ax^{\lambda}$, exponential-- $g(x) = ab^x$, and sigmoid-- $g(x) = \frac{L}{1+e^{k(x-x_0)}}$. 

\textbf{Interaction among the Factors.} We use aggregate functions to capture the interaction among multiple factors of a given content type. In particular, we consider the pairwise interaction functions listed below. 

\begin{table}[h!]
  \centering
  \begin{tabular}{m{.28\textwidth}c}
%    \hline \textbf{Interaction Type} & \textbf{Contour}\\ \hline
    %\begin{minipage}[t]{5cm}
    \vspace{-5pt}
    \uline{Essential:} Essential factors are both required for content generation, with zero marginal return for a single factor. For a pair of essential factors, content generation is determined by the more limiting factor: $z = min(y_1, y_2)$. This is known as Liebig's law of the minimum.
    %\end{minipage} 
    &
    \begin{minipage}{.17\textwidth}
      \includegraphics[width=\textwidth, height=\textwidth]{Figures/Essential.pdf}
    \end{minipage}
    \\ 
    %\begin{minipage}[t]{5cm}
    \vspace{-5pt}
    \uline{Interactive Essential:} In interactive essential interaction, we get diminishing return (instead of zero return) for a single factor: $z = y_1y_2$. If each factor is consumed using power basis function, i.e., $y_i = ax^\lambda_i$, it captures Cobb-Douglas production function.
    %\end{minipage} 
    &
    \begin{minipage}{.17\textwidth}
      \includegraphics[width=\textwidth, height=.975\textwidth]{Figures/Interactive_Essential.pdf}
    \end{minipage}
    \\
    %\begin{minipage}[t]{5cm}
    \vspace{-5pt}
    \uline{Antagonistic:} For antagonistic factors, content generation is determined solely by the availability of the factor which yields the largest return: $z = max(y_1, y_2)$. This interaction implies that the production process has maximum possible efficiency. 
    %\end{minipage} 
    &
    \begin{minipage}{.17\textwidth}
      \includegraphics[width=\textwidth, height=.975\textwidth]{Figures/Antagonistic.pdf}
    \end{minipage}
    \\
    %\begin{minipage}[t]{5cm}
    \vspace{-5pt}
    \uline{Substitutable:} Factors that can each support production on their own are substitutable relative to each other: $z = w_1y_1 + w_2y_2$. This implies that there exists some equivalence between the two factors. This is similar to the general additive models.
    %\end{minipage} 
    &
    \begin{minipage}{.17\textwidth}
      \includegraphics[width=\textwidth, height=.975\textwidth]{Figures/Substitutable.pdf}
    \end{minipage}
    \\
%    \hline
  \end{tabular}
  \vspace{-\baselineskip}
%  \caption{Pairwise interaction between factors}
  \label{tab:interaction}
\end{table}

\textbf{User to Roles.} 
We predict the number of askers, answerers, and commenters from the number of users. \textcolor{blue}{Yet to add details. Figure: LV plots to show the r-squared distribution for regression.}




\section{Dataset} 
We collected the latest release (September, 2017) of Stack Exchange dataset. This snapshot is a complete archive of all activities in all Stack Exchange. There are 169 sites in our collected dataset. For the purpose of empirical analysis, we only consider the sites that have been active for at least 12 months beyond the ramp up period (site created, but few or no activity). There are 156 such sites. The age of these sites vary from 14 months to 111 months, number of user from 1072 to 547175, number of posts (questions and answers) from 1600 to 1985869. Further, the sites have small overlaps in user base; therefore, we can reasonably argue that the underlying markets are independent. 

In Figure~\ref{fig:dataset} we present letter value plots (in log-scale) to show the distribution of number of users, number of posts, and age for the Stack Exchange markets. 


\iffalse
\begin{wrapfigure}{R}{0.2\textwidth}
\centering
\vspace{-\baselineskip}
\includegraphics[width=0.2\textwidth]{Figures/Dataset_Statistics.pdf}
\caption{Site Statistics}
\vspace{-\baselineskip}
\end{wrapfigure}
\fi

\begin{figure}[hbt]
\vspace{-\baselineskip}
\centering
\includegraphics[scale=0.45]{Figures/Dataset_Statistics.pdf}
\vspace{-\baselineskip}
\caption{The distribution of number of users, number of posts, and age for Stack Exchange markets (in log-scale). The markets vary in all three dimensions.}
\vspace{-\baselineskip}
\label{fig:dataset}
\end{figure}

\section{Evaluating Our Proposed Models}
In this section we identify optimal models (basis and interaction) from three different perspectives: the accuracy of fitting content generation time series observed in our dataset (Section 6.1), the performance of predicting content volume in long run (Section 6.2), and the perplexity of characterizing content generation dynamics at early stage (Section 6.3).

\subsection{Model Fitting}
We fit each variant of production model (basis and interaction) for each content type to the observed content generation time series (monthly granularity) in each Stack Exchange website. Notice that among the different variants of production models, the models using power or exponent basis have a parsimonious set of parameters. For example, answer generation model using power basis function requires only three parameters for interactive essential interaction (See Section 4.2), and four parameters for remaining interaction types. In contrast, answer generation model using sigmoid basis function requires five parameters for interactive essential interaction, and six parameters for remaining interaction types. 

\textbf{Parameter Estimation.} Our parameter learning process has three sequential steps enforced by the content dependency among question, answer and comment; we first learn the best-fit parameters for modeling question, followed by answer, followed by comment. At each step, we use the parameters learnt in earlier steps to generate input.

We restrict some parameters of our production models to be non-negative, e.g., non-negative exponents in power basis. These restrictions are important because the underlying factors positively affect the output. We use the trust-region reflective algorithm to solve our constrained least square optimization problem. The algorithm is appropriate for solving non-linear least squares problems with constraints.

\textbf{Evaluation Method.} We evaluate the overall fitting accuracy using four metrics: root mean square error (RMSE), normalized root mean square Error (NRMSE), explained variance score (EVS), and Akaike information criterion (AIC). Given two series, the observed series for content $w$, $N_w(t)$, and the prediction $\hat{N_w(t)}$ of the series by a model with $k$ parameters, we compute these metrics as follows: RMSE = $\sqrt{\frac{1}{T}\sum_{t=1}^{T}(N_w(t)-\hat{N_w(t)})^2}$; NRMSE = $\frac{RMSE}{max(N_w(t))-min(N_w(t))}$; EVS = $1-\frac{Var(N_w(t)-\hat{N_w(t)})}{Var(N_w(t))}$; AIC = $T*ln(\frac{1}{T}\sum_{t=1}^{T}(N_w(t)-\hat{N_w(t)})^2)+2k$. Among the four metrics, RMSE and NRMSE are error metrics (low value implies good fit), AIC is an information theoretic metric to capture the trade-off between model complexity and goodness-of-fit (low value implies good model), and EVS refers to a model's ability to capture variance in data (high value implies good model). All four metrics are consistent with the non-linear least squares problem. 

\textbf{Fitting Results.} Now, we compare the fitting accuracy of production models for all markets in Stack Exchange, using the four metrics, each summarized via mean, for each content type (question, answer and comment). We found that the models with exponential and sigmoid basis do not fit the data for many Stack Exchange. Accordingly, in Table~\ref{tbl:model_fit}, we only present the results for production models with power basis and different interaction types. Notice that the models with interactive essential interaction outperform the remaining models for all metrics and content types. We performed paired t-tests to determine if the improvements for interactive essential interaction are statistically significant; the results are positive with $p<0.01$.

\begin{table}[ht]
	\vspace{-0.5\baselineskip}
	\caption{The comparison of fitting accuracy of production models (with power basis and different interaction types) for all Stack Exchange. The models with interactive essential interaction outperform the remaining models for all metrics and content types. The improvements for interactive essential interaction are statistically significant---validated through paired t-tests.}
    \vspace{-\baselineskip}
	\label{tbl:model_fit}
	\begin{center}
	\begin{tabular}{llcccc}
    \toprule
    \multirow{2}{*}{Content} & Interaction & Avg. & Avg. & Avg. & Avg.\\
    & Type & RMSE & NRMSE & EVS & AIC\\
    \midrule
    Question & Single Factor & 25.742 & 0.086 & 0.791 & 104.473\\
    \midrule
    \multirow{4}{*}{Answer} & Essential & 70.307 & 0.092 & 0.789 & 208.820\\
    & I. Essential & \textbf{64.624} & \textbf{0.083} & \textbf{0.825} & \textbf{196.395}\\
    & Antagonistic & 72.765 & 0.094 &  0.778 & 210.958\\
    & Substitutable & 68.900 & 0.089 & 0.805 & 207.609\\
    \midrule
    \multirow{4}{*}{Comment} & Essential & 146.644 & 0.084 & 0.833 & 328.245\\
    & I. Essential & \textbf{137.228} & \textbf{0.081} & \textbf{0.845} & \textbf{318.243}\\
    & Antagonistic & 155.969 & 0.088 &  0.818 & 334.118\\
    & Substitutable & 155.433 & 0.089 & 0.820 & 335.102\\
    \bottomrule
	\end{tabular}
	\end{center}
    \vspace{-\baselineskip}
\end{table}

\subsection{Forecasting Content Generation} 
We apply production models with power basis and interactive essential interaction to forecast content volume in long run---one year ahead in future. Specifically, we train a model using the first 12 months, and examine how well the model forecasts content dynamics in next 12 months; with monthly granularity. We validate the forecasting capability by examining the overall prediction error (NRMSE) into the future. 

We summarize the prediction NRMSE for all Stack Exchange using mean and variance--- (i) question: 0.11 (mean), 0.08 (variance); (ii) answer: 0.12 (mean), 0.09 (variance); (iii) comments: 0.11 (mean) 0.10 (variance). We performed these experiments for different time granularity, e.g., week, month, quarter, and found consistent conclusion. We do not report these results for brevity.

\subsection{Parameter Estimation for New Markets} 
To better predict the success or failure of new markets (6-12 months old), we use model parameters learnt from old markets (at least 36 months old) as priors for new markets. We validate the parameter estimation capability by first training models using first 6 months of new market, and the priors based on the nearest (based on user size) old market, and then examining how well the model forecasts content dynamics in remaining months. 

We summarize the prediction NRMSE for new Stack Exchange using mean and variance--- (i) question: 0.1 (mean), 0.12 (variance); (ii) answer: 0.09 (mean), 0.08 (variance); (iii) comments: 0.14 (mean), 0.13 (variance).


\section{Characterizing Knowledge Markets}
In this section we characterize the knowledge markets in Stack Exchange---explaining the best-fit models and their foundations (Section 7.1), revealing two key distributions that control the markets (Section 7.2), and uncovering the stable core that maintains market equilibrium (Section 7.3).

\subsection{Model Interpretation} 
First, we explain the best-fit models found in Section 6.1. We observe that content generation in Stack Exchange markets are best modeled through the combination of power basis and interactive essential interaction. In addition, we found that the best-fit exponents ($\lambda$ parameter in basis $g(x) = ax^\lambda$, where $x$ is a factor) of these models lie between 0 and 1 (inclusive), for all factors of all content types, for all Stack Exchange. 

A model that uses power basis (where exponents lie between 0 and 1) and interactive essential interaction is known as the Cobb-Douglas production function~\cite{wiki}. In its most standard form for production of a single output $z$ with two inputs $x_1$ and $x_2$, the function is: 
$$z = ax_1^{\lambda_1}x_2^{\lambda_2}.$$
Here, coefficient $a$ represents the \emph{total factor productivity}---the portion of output not explained by the amount of inputs used in production~\cite{wiki}. As such, its level is determined by how efficiently the inputs are utilized in production. The exponents $\lambda$s represent the \emph{output elasticity} of the inputs---the percentage change in output that results from the percentage change in a particular input~\cite{wiki}. 

The Cobb-Douglas function provides intuitive explanation for content generation in Stack Exchange markets. In particular, the explanation stands on three phenomena or principles: constant elasticity, diminishing returns, and returns to scale.

\textbf{Constant Elasticity.} In Stack Exchange markets, factors such as user participation and content dependency have \emph{constant elasticity}---percentage increase in any of these inputs will have constant percentage increase in output~\cite{wiki}, as claimed by the corresponding exponents in the model. For example, in \texttt{academia} ($N_A = 6.93N_q^{0.18}U_a^{0.65}$), 1\% increase in number of answerers ($U_a$) leads to 0.65\% increase in number of answers ($N_a$). 

\textbf{Diminishing Returns.} For a particular factor, when the exponent is less than 1, we observe \emph{diminishing returns}---decrease in the marginal (incremental) output as an input is incrementally increased, while the other inputs are kept constant~\cite{wiki}. This \lq law of diminishing returns\rq\ has many interesting implications for the Stack Exchange markets, including the diminishing benefit of having a new participant in a market. For example, in \texttt{academia}, if the number of answerers is 100, then the marginal contribution of a new answerer is $c(101^{0.65} - 100^{0.65}) = 0.129c$, where $c$ is a constant; in contrast, if the number of answerers is 110, then the marginal contribution of a new answerer is $c(111^{0.65} - 110^{0.65}) = 0.125c$. Thus, for answer generation in \texttt{academia}, including a participant when the number of participants (system size) is 110 is likely to be less beneficial compared to including a participant when the system size is 100.

\textbf{Returns to scale.} The knowledge markets in Stack Exchange vary in terms of scale efficiency, as manifested by their \emph{returns to scale}---the increase in output resulting from a proportionate increase in all inputs~\cite{wiki}. If a market has high returns to scale, then greater efficiency is obtained as the market moves from small- to large-scale operations. For example, in \texttt{academia}, for answer generation, the returns to scale is $0.18+0.65=0.83<1$. The market becomes less efficient as answer generation is expanded, requiring more questions and answerers to increase the number of answers by same amount. 

\subsection{Two Key Distributions} 
Next, we discuss two key distributions that control content generation in knowledge markets, namely participant activity and subject POV (perspective). These two distributions induce the three phenomena reported in section 7.1. 

\textbf{Participant Activity.} The distribution of participant activities implicitly drives a market's return in terms of user participation, as manifested by the corresponding exponent. For example, in a hypothetical knowledge market where each answerer contributes equally, the answer generation model should be $N_a = AN_q^{\lambda_1}U_a^{1.0}$. In reality, the distribution of participant activities is a size dependent distribution controlled by the number of participants (system size). As the system size increases, most participants contribute to the head of the distribution (few activities), whereas very few join the tail (many activities). 

We systematically reveal the size dependent distribution for participant activities in three steps. First, we empirically fit power-law distribution to the activities of participants in a month, for each month, for each Stack Exchange. We follow the standard procedure to fit a power-law distribution. We observe that power-law well describe the monthly activity distributions. Second, we plot the exponents of power-law against the number of participants for all observed months in a Stack Exchange, for each Stack Exchange. We observe that for most Stack Exchange power-law exponent decreases as the system size increases. Third, we apply linear regression to reveal the relationship between power-law exponents and system size. We observe that in general power-law exponents are negatively correlated with system size. This negative correlation is strongly visible in big knowledge markets that have at least 500 monthly participants in each month.

In Figure~\ref{fig:sdd} we present empirical evidence of size dependent distribution for answer generation in three markets: \texttt{android}, \texttt{apple}, and \texttt{biology}. We choose these examples to cover three possible visibility of size dependent distribution, as manifested by the correlation between 
power law exponent and system size---strong correlation ($|r^2|>0.5$), moderate correlation($0.3<|r^2|<0.5$), weak correlation ($|r^2|<0.3$).

\begin{figure}[hbt]
\vspace{-0.5\baselineskip}
\centering
\includegraphics[scale=0.38]{Figures/Size_Dependent_Distribution.pdf}
\vspace{-2\baselineskip}
\caption{The visibility of size dependent distribution in \texttt{android} (strong), \texttt{apple} (moderate), and \texttt{biology} (weak). In most markets, the power-law exponent decreases with system size (similar to academia). In other markets, there exists a non-zero correlation between system size and power-law exponent.}
\vspace{-\baselineskip}
\label{fig:sdd}
\end{figure}

\textbf{Subject POV.} The distribution of subject POV implicitly drives a market's return in terms of content dependency, as manifested by the corresponding exponent. Subject POV refers to the number of distinct perspectives on a particular content (primarily question) that imposes a conceptual limit to the number of dependent contents (answers). For example, an open-ended question such as \lq What's your favorite book?\rq\ has many possible answers, whereas a close-ended question such as \lq What's the solution for 3x+5 = 2?\rq\ has a single correct answer. In reality, most questions are neither completely open-ended nor completely closed; however, from an answerer's perspective, there's a diminishing utility in answering a question that already has an answer. This diminishing utility varies from question to question---questions asking for recommendations attract many answers, whereas questions seeking factual information attract few answers. 

\subsection{Uncovering the Stable Core} 
Now, we discuss about the core user community that assist maintaining the Cobb-Douglas models in a knowledge market. The Cobb-Douglas models indicate the presence of dynamic equilibria where the increase or decrease in user community does not affect the models. To this end, we assert that there is a stable user community in each knowledge markets who contribute a large fraction of contents; whereas the remaining users are unstable and contribute a small fraction. This is particularly applicable for contents that require more effort, e.g, answers and comments. 

We systematically reveal the presence of core user community by summarizing the age of users with different levels of contribution for all Stack Exchange. First, we categorize the users within a market using quartiles based on average number of contributions per month. Then, for each user in each market, we compute normalized age by dividing the user's age (number of active months) with the maximum possible age in the market. Next, we summarize the normalized age of users with same levels of contribution for all markets using letter value plots. This allows us to observe the global distribution of normalized age for users with varying levels of monthly contribution. We present these distributions in Figure~\ref{fig:age_vs_contribution}. We observe that there is a clear gap between the distribution of age of users for any two levels of contribution---users in 4th quartile (who contribute a lot) have highest normalized age, followed by 3rd and 2nd quartile. 

\begin{figure}[hbt]
\centering
%\includegraphics[scale=0.38]{Figures/Size_Dependent_Distribution.pdf}
\caption{The global distribution of normalized age for users with varying levels of monthly contribution. The users who contribute most to a market on a monthly basis, also contribute for a l}
\label{fig:age_vs_contribution}
\end{figure}



\section{Diseconomies of Scale}
In this section we discuss the diseconomies of scale that occur in the knowledge markets.

\subsection{Empirical Observation}
Backed by the diminishing returns, Stack Exchange websites undergo diseconomies of scale---the ratio of answers to questions go down with the increase in number of users. We compare the economic curves with logistic curves to justify effectiveness in capturing diseconomies of scale. \textcolor{blue}{Figure: Plot to show answer to question ratio vs number of users fit for economic and logistic curves.}

\subsection{Decline in Health} 
As the health of knowledge markets directly depend on content generation, we investigate the effect of scale on a set of health metrics. \textcolor{blue}{Figure: Plot to show correlation between answer to question ratio and health metrics}

\subsection{Decline in Stability} 
As the stability of knowledge markets depend on user participation, we investigate the effect of scale on stability. \textcolor{blue}{Figure: Plot to show stability as a function of economic parameters/size/etc.}


\section{Discussion}
In this section we discuss the implications of our research (Section 9.1), the limitations of our work (Section 9.2), and  complementary models (Section 9.3).

\subsection{Implications}
Our work draws from, and has implications for several research threads.

\textbf{Power Law of Participation.} Ross Mayfield coined the term \lq Power Law of Participation\rq\---when a small number of community members participate in high engagement activities, while the larger community participate in low threshold activities. We observe power-law of participation in Stack Exchange markets. Stack Exchange supports varieties of activities ranging from low-threshold (e.g., voting) to high-engagement (e.g., collaborative editing, and linking similar questions); with a small fraction of users participating in high-engagement activities. 

We assert that both low-threshold and high-engagement activities are required for a knowledge market's survival, and should proportionately increase with the increase in number of participants. However, in reality, for most knowledge markets, the user community contributing high-engagement activities does not scale with the system size---which creates a gap between market supply and demand for high-engagement activities, and consequently affects market health. 

%He observed that for communities such as Wikipedia, Flickr, and Digg, several high engagement activities (e.g., Wikipedia editing) require collaboration among users, whereas low engagement activities are somewhat unattached.

\textbf{Controllability.}
In a complex network, controllability refers to the identification of the driver nodes that can guide the system’s entire dynamics. Controllability is a function of degree distribution 
%\subsection{Micro-foundations}

\subsection{Limitations}
Now, we discuss several limitations of our work. First, the economic production models do not account for user growth. While there exists several user growth models for two-sided markets, membership based websites, and online social networks, it would be useful to introduce an economic user growth model that properly complements the production models. A potential direction in this research is to apply Malthusian growth model. Second, the production models inherit the fundamental assumptions of macroeconomics such as an aggregate is homogeneous (without looking into its internal composition), and aggregates are functionally related etc. It would be useful to study these assumptions in detail for knowledge markets. Third, 


\subsection{Complementary Models} 
We considered several alternative models to comprehend the content generation dynamics in knowledge markets. Our first attempt was to model user content generation as \lq self-exciting\rq\ point process. In this attempt, we designed and implemented several variants of Hawkes process. Our second attempt was to model user content generation using \lq stage-structured\rq\ projection matrix. In this attempt, we build variants of Leftkovitch matrix. 

\iffalse
\textbf{Hawkes Process.} The Hawkes process is a mathematical model for self-exciting processes that models a sequence of arrivals of some event over time, e.g., natural disasters, gang violence, trade orders.  Each arrival excites the process in the sense that the chance of a subsequent arrival is increased for some time period after the initial arrival. 

Content generation in Stack Exchange websites can be conceptualized as Hawkes process with multiple event types. Here, arrival of certain type of event (e.g., questions) excites the arrival of other types of event (e.g., answers, comments). 

A univariate Hawkes process is defined to be a self-exciting temporal point process $N$ whose conditional intensity function $\lambda = \lambda(t)$ is defined to be

 $$\lambda(t) = \mu(t)+\sum_{i:\tau_i<t}g(t-\tau_i),$$
 
where $\mu(t)$ is the background rate of the process  $N$, where $\tau_i$ are the points/events in time occurring prior to time t, and where $g$ is a function which governs the clustering density of N. 

The multi-dimensional Hawkes process is defined by a $U$-dimensional point process $N_t^u, u =1, . . . , U$, with the conditional intensity for the $u$-th dimension expressed as follows:
$$\lambda_u(t) = \mu_u + \sum_{i:\tau_i<t} g_{uu_i}(t-\tau_i),$$

where $\mu_u \ge 0$ is the base intensity for the $u$-th Hawkes process. The kernel $g_{uu^\prime}(t) \ge 0$ captures the mutually exciting property between the $u$-th and $u^\prime$-th dimension. Intuitively, it captures the dynamics of influence of events occurred in the $u^\prime$-th dimension to the $u$-th dimension. Larger value of $g_{uu^\prime}(t)$  indicates that events in $u^\prime$-th dimension are more likely to trigger a event in the $u$-th dimension after a time interval $t$.

\textbf{Leftkovitch Matrix.} \textcolor{red}{A brief description of the Leftkovitch Matrix based models.}
\fi



\section{Conclusion}

% Chase's version
In this paper, we examined CQA websites on the StackExchange platform through an economic lens by modeling them as knowledge markets. We analyzed a set of basis functions (the functional form of how an input factor, such as the number of available answerers, contributes to the overall output of the network) and interaction mechanisms (how the input  factors interact with each other) to capture the content generation dynamics in knowledge markets. The resulting best-fit model, Cobb-Douglas, predicts the production of content with high accuracy. In addition, we showed that the model provides intuitive explanations for content generation in Stack Exchange markets. Namely, \begin{enumerate*}
    \item factors such as user participation and content dependency have \emph{constant elasticity}, meaning that a percentage increase in any of these inputs will result in a constant percentage increase in output;
    \item input factors exhibit \emph{diminishing returns} in that there is a decrease in the marginal (incremental) output (content production) as an input (e.g. number of people who answer) is incrementally increased while holding the other inputs constant; and
    \item the efficiency of markets varies as manifest by their \emph{returns to scale}---the increase in output resulting from a proportionate increase in all inputs
\end{enumerate*}. Finally, we explore the implications of the Cobb-Douglas economic model by showing the presence of diseconomies of scale in terms of the production of content, several measures of network health, and finally the exchangability of users in the network. We conclude that there is more to a successful CQA network than merely the total number of active users. Blindly increasing network sizes and ignoring these potential diseconomies of scale is unwise if we wish to sustain healthy knowledge markets.

% Himel's version
%In this paper, we examine CQA websites in Stack Exchange through an economic lens by modeling them as knowledge markets. We analyze a set of basis functions (the function form of how an input contributes to output) and interaction mechanisms (how the inputs interact with each other) to capture content generation dynamics in knowledge markets. The resultant best-fit model, Cobb-Douglas, predicts the production of content with high accuracy. In addition, the model provides intuitive explanation for content generation in Stack Exchange markets. Specifically, it shows that factors of content generation such as user participation and content dependency have \emph{constant elasticity}; in many markets, factors exhibit \emph{diminishing returns}; markets vary according to their \emph{returns to scale}; and finally many markets exhibit \emph{diseconomies of scale}. 


%\documentclass[sigconf]{acmart}

\usepackage{booktabs} % For formal tables
% My packages
\usepackage{tikz}
\usetikzlibrary{bayesnet}
\usepackage[normalem]{ulem}
\usepackage{wrapfig}
\usepackage{xcolor}
\usepackage{array}
\newcommand{\hs}[1]{\textcolor{red}{Hari: #1}}

% Copyright
%\setcopyright{none}
%\setcopyright{acmcopyright}
%\setcopyright{acmlicensed}
%\setcopyright{rightsretained}
%\setcopyright{usgov}
%\setcopyright{usgovmixed}
%\setcopyright{cagov}
%\setcopyright{cagovmixed}


% DOI
%\acmDOI{10.475/123_4}

% ISBN
%\acmISBN{123-4567-24-567/08/06}

%Conference
\iffalse
\acmConference[WOODSTOCK'97]{ACM Woodstock conference}{July 1997}{El
  Paso, Texas USA}
\acmYear{1997}
\fi

%\copyrightyear{2016}

%\acmPrice{15.00}


\begin{document}
\title[Too Big to Succeed]{Too Big to Succeed: Understanding Successes and Failures at Scale in Knowledge Markets}

\iffalse
\titlenote{Produces the permission block, and
  copyright information}
\subtitle{Extended Abstract}
\subtitlenote{The full version of the author's guide is available as
  \texttt{acmart.pdf} document}
\fi

\author{Anonymous Author}
\affiliation{%
  \institution{Anonymous Institution}
  \city{City, Country}
}
\email{e-mail}

\iffalse
\author{Ben Trovato}
\authornote{Dr.~Trovato insisted his name be first.}
\orcid{1234-5678-9012}
\affiliation{%
  \institution{Institute for Clarity in Documentation}
  \streetaddress{P.O. Box 1212}
  \city{Dublin}
  \state{Ohio}
  \postcode{43017-6221}
}
\email{trovato@corporation.com}

\author{G.K.M. Tobin}
\authornote{The secretary disavows any knowledge of this author's actions.}
\affiliation{%
  \institution{Institute for Clarity in Documentation}
  \streetaddress{P.O. Box 1212}
  \city{Dublin}
  \state{Ohio}
  \postcode{43017-6221}
}
\email{webmaster@marysville-ohio.com}

\author{Lars Th{\o}rv{\"a}ld}
\authornote{This author is the
  one who did all the really hard work.}
\affiliation{%
  \institution{The Th{\o}rv{\"a}ld Group}
  \streetaddress{1 Th{\o}rv{\"a}ld Circle}
  \city{Hekla}
  \country{Iceland}}
\email{larst@affiliation.org}

\author{Lawrence P. Leipuner}
\affiliation{
  \institution{Brookhaven Laboratories}
  \streetaddress{P.O. Box 5000}}
\email{lleipuner@researchlabs.org}

\author{Sean Fogarty}
\affiliation{%
  \institution{NASA Ames Research Center}
  \city{Moffett Field}
  \state{California}
  \postcode{94035}}
\email{fogartys@amesres.org}

\author{Charles Palmer}
\affiliation{%
  \institution{Palmer Research Laboratories}
  \streetaddress{8600 Datapoint Drive}
  \city{San Antonio}
  \state{Texas}
  \postcode{78229}}
\email{cpalmer@prl.com}

\author{John Smith}
\affiliation{\institution{The Th{\o}rv{\"a}ld Group}}
\email{jsmith@affiliation.org}

\author{Julius P.~Kumquat}
\affiliation{\institution{The Kumquat Consortium}}
\email{jpkumquat@consortium.net}
\fi

% The default list of authors is too long for headers}
%\renewcommand{\shortauthors}{B. Trovato et al.}

\iffalse
\begin{abstract}
This paper provides a sample of a \LaTeX\ document which conforms,
somewhat loosely, to the formatting guidelines for
ACM SIG Proceedings.\footnote{This is an abstract footnote}
\end{abstract}

%
% The code below should be generated by the tool at
% http://dl.acm.org/ccs.cfm
% Please copy and paste the code instead of the example below.
%
\begin{CCSXML}
<ccs2012>
 <concept>
  <concept_id>10010520.10010553.10010562</concept_id>
  <concept_desc>Computer systems organization~Embedded systems</concept_desc>
  <concept_significance>500</concept_significance>
 </concept>
 <concept>
  <concept_id>10010520.10010575.10010755</concept_id>
  <concept_desc>Computer systems organization~Redundancy</concept_desc>
  <concept_significance>300</concept_significance>
 </concept>
 <concept>
  <concept_id>10010520.10010553.10010554</concept_id>
  <concept_desc>Computer systems organization~Robotics</concept_desc>
  <concept_significance>100</concept_significance>
 </concept>
 <concept>
  <concept_id>10003033.10003083.10003095</concept_id>
  <concept_desc>Networks~Network reliability</concept_desc>
  <concept_significance>100</concept_significance>
 </concept>
</ccs2012>
\end{CCSXML}

\ccsdesc[500]{Computer systems organization~Embedded systems}
\ccsdesc[300]{Computer systems organization~Redundancy}
\ccsdesc{Computer systems organization~Robotics}
\ccsdesc[100]{Networks~Network reliability}


\keywords{ACM proceedings, \LaTeX, text tagging}
\fi


\maketitle


%\documentclass[sigconf]{acmart}

\usepackage{booktabs} % For formal tables
% My packages
\usepackage{tikz}
\usetikzlibrary{bayesnet}
\usepackage[normalem]{ulem}
\usepackage{wrapfig}
\usepackage{xcolor}
\usepackage{array}
\newcommand{\hs}[1]{\textcolor{red}{Hari: #1}}

% Copyright
%\setcopyright{none}
%\setcopyright{acmcopyright}
%\setcopyright{acmlicensed}
%\setcopyright{rightsretained}
%\setcopyright{usgov}
%\setcopyright{usgovmixed}
%\setcopyright{cagov}
%\setcopyright{cagovmixed}


% DOI
%\acmDOI{10.475/123_4}

% ISBN
%\acmISBN{123-4567-24-567/08/06}

%Conference
\iffalse
\acmConference[WOODSTOCK'97]{ACM Woodstock conference}{July 1997}{El
  Paso, Texas USA}
\acmYear{1997}
\fi

%\copyrightyear{2016}

%\acmPrice{15.00}


\begin{document}
\title[Too Big to Succeed]{Too Big to Succeed: Understanding Successes and Failures at Scale in Knowledge Markets}

\iffalse
\titlenote{Produces the permission block, and
  copyright information}
\subtitle{Extended Abstract}
\subtitlenote{The full version of the author's guide is available as
  \texttt{acmart.pdf} document}
\fi

\author{Anonymous Author}
\affiliation{%
  \institution{Anonymous Institution}
  \city{City, Country}
}
\email{e-mail}

\iffalse
\author{Ben Trovato}
\authornote{Dr.~Trovato insisted his name be first.}
\orcid{1234-5678-9012}
\affiliation{%
  \institution{Institute for Clarity in Documentation}
  \streetaddress{P.O. Box 1212}
  \city{Dublin}
  \state{Ohio}
  \postcode{43017-6221}
}
\email{trovato@corporation.com}

\author{G.K.M. Tobin}
\authornote{The secretary disavows any knowledge of this author's actions.}
\affiliation{%
  \institution{Institute for Clarity in Documentation}
  \streetaddress{P.O. Box 1212}
  \city{Dublin}
  \state{Ohio}
  \postcode{43017-6221}
}
\email{webmaster@marysville-ohio.com}

\author{Lars Th{\o}rv{\"a}ld}
\authornote{This author is the
  one who did all the really hard work.}
\affiliation{%
  \institution{The Th{\o}rv{\"a}ld Group}
  \streetaddress{1 Th{\o}rv{\"a}ld Circle}
  \city{Hekla}
  \country{Iceland}}
\email{larst@affiliation.org}

\author{Lawrence P. Leipuner}
\affiliation{
  \institution{Brookhaven Laboratories}
  \streetaddress{P.O. Box 5000}}
\email{lleipuner@researchlabs.org}

\author{Sean Fogarty}
\affiliation{%
  \institution{NASA Ames Research Center}
  \city{Moffett Field}
  \state{California}
  \postcode{94035}}
\email{fogartys@amesres.org}

\author{Charles Palmer}
\affiliation{%
  \institution{Palmer Research Laboratories}
  \streetaddress{8600 Datapoint Drive}
  \city{San Antonio}
  \state{Texas}
  \postcode{78229}}
\email{cpalmer@prl.com}

\author{John Smith}
\affiliation{\institution{The Th{\o}rv{\"a}ld Group}}
\email{jsmith@affiliation.org}

\author{Julius P.~Kumquat}
\affiliation{\institution{The Kumquat Consortium}}
\email{jpkumquat@consortium.net}
\fi

% The default list of authors is too long for headers}
%\renewcommand{\shortauthors}{B. Trovato et al.}

\iffalse
\begin{abstract}
This paper provides a sample of a \LaTeX\ document which conforms,
somewhat loosely, to the formatting guidelines for
ACM SIG Proceedings.\footnote{This is an abstract footnote}
\end{abstract}

%
% The code below should be generated by the tool at
% http://dl.acm.org/ccs.cfm
% Please copy and paste the code instead of the example below.
%
\begin{CCSXML}
<ccs2012>
 <concept>
  <concept_id>10010520.10010553.10010562</concept_id>
  <concept_desc>Computer systems organization~Embedded systems</concept_desc>
  <concept_significance>500</concept_significance>
 </concept>
 <concept>
  <concept_id>10010520.10010575.10010755</concept_id>
  <concept_desc>Computer systems organization~Redundancy</concept_desc>
  <concept_significance>300</concept_significance>
 </concept>
 <concept>
  <concept_id>10010520.10010553.10010554</concept_id>
  <concept_desc>Computer systems organization~Robotics</concept_desc>
  <concept_significance>100</concept_significance>
 </concept>
 <concept>
  <concept_id>10003033.10003083.10003095</concept_id>
  <concept_desc>Networks~Network reliability</concept_desc>
  <concept_significance>100</concept_significance>
 </concept>
</ccs2012>
\end{CCSXML}

\ccsdesc[500]{Computer systems organization~Embedded systems}
\ccsdesc[300]{Computer systems organization~Redundancy}
\ccsdesc{Computer systems organization~Robotics}
\ccsdesc[100]{Networks~Network reliability}


\keywords{ACM proceedings, \LaTeX, text tagging}
\fi


\maketitle


%\documentclass[sigconf]{acmart}

\usepackage{booktabs} % For formal tables
% My packages
\usepackage{tikz}
\usetikzlibrary{bayesnet}
\usepackage[normalem]{ulem}
\usepackage{wrapfig}
\usepackage{xcolor}
\usepackage{array}
\newcommand{\hs}[1]{\textcolor{red}{Hari: #1}}

% Copyright
%\setcopyright{none}
%\setcopyright{acmcopyright}
%\setcopyright{acmlicensed}
%\setcopyright{rightsretained}
%\setcopyright{usgov}
%\setcopyright{usgovmixed}
%\setcopyright{cagov}
%\setcopyright{cagovmixed}


% DOI
%\acmDOI{10.475/123_4}

% ISBN
%\acmISBN{123-4567-24-567/08/06}

%Conference
\iffalse
\acmConference[WOODSTOCK'97]{ACM Woodstock conference}{July 1997}{El
  Paso, Texas USA}
\acmYear{1997}
\fi

%\copyrightyear{2016}

%\acmPrice{15.00}


\begin{document}
\title[Too Big to Succeed]{Too Big to Succeed: Understanding Successes and Failures at Scale in Knowledge Markets}

\iffalse
\titlenote{Produces the permission block, and
  copyright information}
\subtitle{Extended Abstract}
\subtitlenote{The full version of the author's guide is available as
  \texttt{acmart.pdf} document}
\fi

\author{Anonymous Author}
\affiliation{%
  \institution{Anonymous Institution}
  \city{City, Country}
}
\email{e-mail}

\iffalse
\author{Ben Trovato}
\authornote{Dr.~Trovato insisted his name be first.}
\orcid{1234-5678-9012}
\affiliation{%
  \institution{Institute for Clarity in Documentation}
  \streetaddress{P.O. Box 1212}
  \city{Dublin}
  \state{Ohio}
  \postcode{43017-6221}
}
\email{trovato@corporation.com}

\author{G.K.M. Tobin}
\authornote{The secretary disavows any knowledge of this author's actions.}
\affiliation{%
  \institution{Institute for Clarity in Documentation}
  \streetaddress{P.O. Box 1212}
  \city{Dublin}
  \state{Ohio}
  \postcode{43017-6221}
}
\email{webmaster@marysville-ohio.com}

\author{Lars Th{\o}rv{\"a}ld}
\authornote{This author is the
  one who did all the really hard work.}
\affiliation{%
  \institution{The Th{\o}rv{\"a}ld Group}
  \streetaddress{1 Th{\o}rv{\"a}ld Circle}
  \city{Hekla}
  \country{Iceland}}
\email{larst@affiliation.org}

\author{Lawrence P. Leipuner}
\affiliation{
  \institution{Brookhaven Laboratories}
  \streetaddress{P.O. Box 5000}}
\email{lleipuner@researchlabs.org}

\author{Sean Fogarty}
\affiliation{%
  \institution{NASA Ames Research Center}
  \city{Moffett Field}
  \state{California}
  \postcode{94035}}
\email{fogartys@amesres.org}

\author{Charles Palmer}
\affiliation{%
  \institution{Palmer Research Laboratories}
  \streetaddress{8600 Datapoint Drive}
  \city{San Antonio}
  \state{Texas}
  \postcode{78229}}
\email{cpalmer@prl.com}

\author{John Smith}
\affiliation{\institution{The Th{\o}rv{\"a}ld Group}}
\email{jsmith@affiliation.org}

\author{Julius P.~Kumquat}
\affiliation{\institution{The Kumquat Consortium}}
\email{jpkumquat@consortium.net}
\fi

% The default list of authors is too long for headers}
%\renewcommand{\shortauthors}{B. Trovato et al.}

\iffalse
\begin{abstract}
This paper provides a sample of a \LaTeX\ document which conforms,
somewhat loosely, to the formatting guidelines for
ACM SIG Proceedings.\footnote{This is an abstract footnote}
\end{abstract}

%
% The code below should be generated by the tool at
% http://dl.acm.org/ccs.cfm
% Please copy and paste the code instead of the example below.
%
\begin{CCSXML}
<ccs2012>
 <concept>
  <concept_id>10010520.10010553.10010562</concept_id>
  <concept_desc>Computer systems organization~Embedded systems</concept_desc>
  <concept_significance>500</concept_significance>
 </concept>
 <concept>
  <concept_id>10010520.10010575.10010755</concept_id>
  <concept_desc>Computer systems organization~Redundancy</concept_desc>
  <concept_significance>300</concept_significance>
 </concept>
 <concept>
  <concept_id>10010520.10010553.10010554</concept_id>
  <concept_desc>Computer systems organization~Robotics</concept_desc>
  <concept_significance>100</concept_significance>
 </concept>
 <concept>
  <concept_id>10003033.10003083.10003095</concept_id>
  <concept_desc>Networks~Network reliability</concept_desc>
  <concept_significance>100</concept_significance>
 </concept>
</ccs2012>
\end{CCSXML}

\ccsdesc[500]{Computer systems organization~Embedded systems}
\ccsdesc[300]{Computer systems organization~Redundancy}
\ccsdesc{Computer systems organization~Robotics}
\ccsdesc[100]{Networks~Network reliability}


\keywords{ACM proceedings, \LaTeX, text tagging}
\fi


\maketitle


%\input{outline}


\section{Introduction}

\section{Related Work}

\section{Problem Formulation}
In this section we describe the desiderata for designing a model to understand the successes/failures of knowledge markets.

\section{Modeling Knowledge Markets}
In this section we introduce production models to capture content generation dynamics in real-world knowledge markets. We first draw an analogy between economic production and content generation (Section 4.1), and then report the content generation factors in knowledge markets (Section 4.2). Next, we concentrate on the knowledge markets in Stack Exchange networks---presenting production models to capture content generation dynamics for different content types (Section 4.3).
\subsection{Production Analogy}
We conceptualize content generation in knowledge markets as economic production.
\subsection{Factors of Content Generation}
We recognize the key factors of content generation in knowledge markets.
\subsection{Modeling Markets in Stack Exchange}
Now, we concentrate on modeling the knowledge markets in Stack Exchange, where each market primarily generates three types of contents: question, answer, and comment.

\section{Dataset}
We collected the latest release (September, 2017) of Stack Exchange dataset.

\section{Evaluating Our Proposed Models}
In this section we examine our proposed models from three different perspectives: the accuracy of fitting content generation time series observed in our dataset (Section 6.1), the performance of predicting content volume in long and short run (Section 6.2), and the perplexity of characterizing content generation dynamics at early stage (Section 6.3).
\subsection{Model Fitting}
We fit each variant of production model for each content type to the observed time series in each Stack Exchange website.
\subsection{Forecasting Content Generation}
We apply the best-fit production models to predict content volume in long and short run.
\subsection{Parameter Estimation for New Websites}
We use parameters learnt from old Stack Exchange websites as priors for new Stack Exchange websites.

\section{Characterizing Knowledge Markets}
In this section we characterize the knowledge markets in Stack Exchange---explaining the best-fit models and their foundations (Section 7.1), revealing two key distributions that control the markets (Section 7.2), and uncovering the stable core that maintains market equilibrium (Section 7.3).
\subsection{Model Interpretation}
First, we explain the best-fit models found in Section 6.1.
\subsection{Two Key Distributions}
Next, we discuss two key distributions that control content generation in knowledge markets, namely participant activity and subject POV (perspective).
\subsection{Uncovering the Stable Core}
Now, we show the presence of a stable core of users that control the dynamic market equilibrium hypothesized by the Cobb-Douglas function.

\section{Diseconomies of Scale}
In this section we discuss the diseconomies of scale that occur in the knowledge markets.
\subsection{Empirical Observation}
Backed by the diminishing returns, Stack Exchange websites undergo diseconomies of scale---the ratio of answers to questions go down with the increase in number of users.
\subsection{Decline in Health}
As the health of knowledge markets directly depend on content generation, we investigate the effect of scale on a set of health metrics.
\subsection{Decline in Stability}
As the stability of knowledge markets depend on user participation, we investigate the effect of scale on stability.

\section{Discussion}
In this section we discuss the analytical findings of our model (Section 8.1) along with a couple of complementary and alternative models (Section 8.2).
\subsection{Analytical Findings}
We present several analytical findings that have implications for different aspects of knowledge markets.
\subsection{Complementary and Alternative Models}
We consider several alternative models to comprehend the content generation dynamics in knowledge markets.

\section{Conclusion}
\bibliographystyle{ACM-Reference-Format}
\bibliography{sigproc}

\end{document}



\section{Introduction}

\section{Related Work}

\section{Problem Formulation}
In this section we describe the desiderata for designing a model to understand the successes/failures of knowledge markets.

\section{Modeling Knowledge Markets}
In this section we introduce production models to capture content generation dynamics in real-world knowledge markets. We first draw an analogy between economic production and content generation (Section 4.1), and then report the content generation factors in knowledge markets (Section 4.2). Next, we concentrate on the knowledge markets in Stack Exchange networks---presenting production models to capture content generation dynamics for different content types (Section 4.3).
\subsection{Production Analogy}
We conceptualize content generation in knowledge markets as economic production.
\subsection{Factors of Content Generation}
We recognize the key factors of content generation in knowledge markets.
\subsection{Modeling Markets in Stack Exchange}
Now, we concentrate on modeling the knowledge markets in Stack Exchange, where each market primarily generates three types of contents: question, answer, and comment.

\section{Dataset}
We collected the latest release (September, 2017) of Stack Exchange dataset.

\section{Evaluating Our Proposed Models}
In this section we examine our proposed models from three different perspectives: the accuracy of fitting content generation time series observed in our dataset (Section 6.1), the performance of predicting content volume in long and short run (Section 6.2), and the perplexity of characterizing content generation dynamics at early stage (Section 6.3).
\subsection{Model Fitting}
We fit each variant of production model for each content type to the observed time series in each Stack Exchange website.
\subsection{Forecasting Content Generation}
We apply the best-fit production models to predict content volume in long and short run.
\subsection{Parameter Estimation for New Websites}
We use parameters learnt from old Stack Exchange websites as priors for new Stack Exchange websites.

\section{Characterizing Knowledge Markets}
In this section we characterize the knowledge markets in Stack Exchange---explaining the best-fit models and their foundations (Section 7.1), revealing two key distributions that control the markets (Section 7.2), and uncovering the stable core that maintains market equilibrium (Section 7.3).
\subsection{Model Interpretation}
First, we explain the best-fit models found in Section 6.1.
\subsection{Two Key Distributions}
Next, we discuss two key distributions that control content generation in knowledge markets, namely participant activity and subject POV (perspective).
\subsection{Uncovering the Stable Core}
Now, we show the presence of a stable core of users that control the dynamic market equilibrium hypothesized by the Cobb-Douglas function.

\section{Diseconomies of Scale}
In this section we discuss the diseconomies of scale that occur in the knowledge markets.
\subsection{Empirical Observation}
Backed by the diminishing returns, Stack Exchange websites undergo diseconomies of scale---the ratio of answers to questions go down with the increase in number of users.
\subsection{Decline in Health}
As the health of knowledge markets directly depend on content generation, we investigate the effect of scale on a set of health metrics.
\subsection{Decline in Stability}
As the stability of knowledge markets depend on user participation, we investigate the effect of scale on stability.

\section{Discussion}
In this section we discuss the analytical findings of our model (Section 8.1) along with a couple of complementary and alternative models (Section 8.2).
\subsection{Analytical Findings}
We present several analytical findings that have implications for different aspects of knowledge markets.
\subsection{Complementary and Alternative Models}
We consider several alternative models to comprehend the content generation dynamics in knowledge markets.

\section{Conclusion}
\bibliographystyle{ACM-Reference-Format}
\bibliography{sigproc}

\end{document}



\section{Introduction}

\section{Related Work}

\section{Problem Formulation}
In this section we describe the desiderata for designing a model to understand the successes/failures of knowledge markets.

\section{Modeling Knowledge Markets}
In this section we introduce production models to capture content generation dynamics in real-world knowledge markets. We first draw an analogy between economic production and content generation (Section 4.1), and then report the content generation factors in knowledge markets (Section 4.2). Next, we concentrate on the knowledge markets in Stack Exchange networks---presenting production models to capture content generation dynamics for different content types (Section 4.3).
\subsection{Production Analogy}
We conceptualize content generation in knowledge markets as economic production.
\subsection{Factors of Content Generation}
We recognize the key factors of content generation in knowledge markets.
\subsection{Modeling Markets in Stack Exchange}
Now, we concentrate on modeling the knowledge markets in Stack Exchange, where each market primarily generates three types of contents: question, answer, and comment.

\section{Dataset}
We collected the latest release (September, 2017) of Stack Exchange dataset.

\section{Evaluating Our Proposed Models}
In this section we examine our proposed models from three different perspectives: the accuracy of fitting content generation time series observed in our dataset (Section 6.1), the performance of predicting content volume in long and short run (Section 6.2), and the perplexity of characterizing content generation dynamics at early stage (Section 6.3).
\subsection{Model Fitting}
We fit each variant of production model for each content type to the observed time series in each Stack Exchange website.
\subsection{Forecasting Content Generation}
We apply the best-fit production models to predict content volume in long and short run.
\subsection{Parameter Estimation for New Websites}
We use parameters learnt from old Stack Exchange websites as priors for new Stack Exchange websites.

\section{Characterizing Knowledge Markets}
In this section we characterize the knowledge markets in Stack Exchange---explaining the best-fit models and their foundations (Section 7.1), revealing two key distributions that control the markets (Section 7.2), and uncovering the stable core that maintains market equilibrium (Section 7.3).
\subsection{Model Interpretation}
First, we explain the best-fit models found in Section 6.1.
\subsection{Two Key Distributions}
Next, we discuss two key distributions that control content generation in knowledge markets, namely participant activity and subject POV (perspective).
\subsection{Uncovering the Stable Core}
Now, we show the presence of a stable core of users that control the dynamic market equilibrium hypothesized by the Cobb-Douglas function.

\section{Diseconomies of Scale}
In this section we discuss the diseconomies of scale that occur in the knowledge markets.
\subsection{Empirical Observation}
Backed by the diminishing returns, Stack Exchange websites undergo diseconomies of scale---the ratio of answers to questions go down with the increase in number of users.
\subsection{Decline in Health}
As the health of knowledge markets directly depend on content generation, we investigate the effect of scale on a set of health metrics.
\subsection{Decline in Stability}
As the stability of knowledge markets depend on user participation, we investigate the effect of scale on stability.

\section{Discussion}
In this section we discuss the analytical findings of our model (Section 8.1) along with a couple of complementary and alternative models (Section 8.2).
\subsection{Analytical Findings}
We present several analytical findings that have implications for different aspects of knowledge markets.
\subsection{Complementary and Alternative Models}
We consider several alternative models to comprehend the content generation dynamics in knowledge markets.

\section{Conclusion}
\bibliographystyle{ACM-Reference-Format}
\bibliography{sigproc}

\end{document}


\bibliographystyle{ACM-Reference-Format}
\bibliography{sigproc}

\end{document}
