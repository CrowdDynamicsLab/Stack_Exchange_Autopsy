Interacting population based models have been successfully applied in a variety of domains to model population growth. In recent times, there has been an increasing interest in applying such models to explain the growth/decline in online social networks \cite{Ribeiro2014,Kumar2010}. While these existing models have been fairly successful in capturing population growth, they have some fundamental limitations such as incorporating largely unobservable factor(s), lack of interpretation in explaining growth, not incorporating observable content factors. %artificial segmentation of population \cite{Kumar2010}

Kumar et al. \cite{Kumar2010} introduced a two-sided (two types of users: blue and green) market model based on the hypothesis that a blue user decides to join a platform based on the platforms' current proportion of green users and vice versa. However, in most social networks, only a small fraction of user community is visible to a potential user, and the proportion of a particular type of user in the network remains largely unobservable. 

Ribeiro et al. \cite{Ribeiro2014} proposed a daily active user (DAU) prediction model based on a set of reaction-diffusion-decay equations describing the interactions between active members, inactive members, and not-yet-members. The proposed model captures the number of daily active users using a set of growth parameters. However, these growth parameters provide little insight in understanding the actual reasons behind the growth/decline in DAU. 

We observe that while population interaction can model population growth in certain social networks, it provides little insight in explaining growth. We hypothesize that population content interaction is a more effective means to model and explain growth in both population and content. Growth models based on population content interaction can reveal the interdependency between the two, i.e., how growth/decline in one affects the other. Such insights are valuable for content based networks, in particular Q\&A networks. In this paper, we focus on jointly modeling population and growth in Stack Exchange websites, revealing the interdependency between the two.

\begin{table*}[hbt]
	\centering
	\begin{tabular}{|l|l|}
	\hline
	\textbf{Related work} & \textbf{Summary}\\ \hline
	Network effect adoption models & individual rationality and adoption costs and utilities\\
	[1][2][3][4][5][6][7][8]& are modeled in a game-theoretic framework.\\ \hline
	
	 \end{tabular}
    \caption{Summary of related work}
\end{table*}