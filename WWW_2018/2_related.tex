\section{Related Work}
\textbf{Modeling Activity Dynamics.} There has been a number of works on modeling activity dynamics in online platforms~\cite{wu2011, anderson2012, walk2016}. Notably, Walk et al. modeled user-level (micro-scale) activity dynamics in Stack Exchange using two factors: intrinsic activity decay, and positive peer influence~\cite{walk2016}. While this model can mimic user-level activity dynamics, it can not be used to study content-driven platform success or failure for following reasons: (i) it does not reveal insights on the collective platform dynamics; (ii) it does not concentrate on the eventual success or failure of a platform. Wu et al. proposed a discrete generalized beta distribution (DGBD) model that reveals insights on the collective dynamics of an online platform---notably, the concept of size-dependent distribution.

\textbf{Successes and Failures of Networks.} Successes and failures of networks have been widely studied from user growth perspective~\cite{Kumar2006, Backstrom2006, kairam2012, zang2016}. Notably, Backstrom et al. studied the mechanisms of how users join communities in a social network~\cite{Backstrom2006};  Kairam et al. examined diffusion (growth via social ties) and non-diffusion (growth without social ties) process to design models that predict the longetivity of social groups~\cite{kairam2012}. These works, however, do not model active user growth. To capture active user growth, Ribeiro et al. proposed a daily active user (DAU) prediction model for membership based websites~\cite{Ribeiro2014}; the model classifies these websites as sustainable and unsustainable.

\textbf{Modeling CQA Websites.} 



\textbf{Modeling Markets.} 

