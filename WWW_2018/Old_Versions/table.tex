\iffalse
\begin{figure}
  \centering
  \tikz{ %
    \node[latent] (alpha) {$\alpha$} ; %
    \node[latent, right=of alpha] (theta) {$\theta$} ; %
    \node[latent, right=of theta] (z) {z} ; %
    \node[latent, above=of z] (beta) {$\beta$} ; %
    \node[obs, right=of z] (w) {w} ; %
    \plate[inner sep=0.25cm, xshift=-0.12cm, yshift=0.12cm] {plate1} {(z) (w)} {N}; %
    \plate[inner sep=0.25cm, xshift=-0.12cm, yshift=0.12cm] {plate2} {(theta) (plate1)} {M}; %
    \edge {alpha} {theta} ; %
    \edge {theta} {z} ; %
    \edge {z,beta} {w} ; %
  }
\end{figure}
\fi

\begin{figure}
  \centering
  \tikz{ %
  	\node[obs] (answerers) {$U_a$} ; %
    \node[obs, right=of answerers] (answers) {$N_a$} ; %
    \node[obs, right=of answers] (comments) {$N_c$} ; %
    \node[obs, above=of answerers] (askers) {$U_q$} ; %
    \node[obs, right=of askers] (questions) {$N_q$} ; %
    \node[obs, right=of questions] (commenters) {$U_c$} ; %
    
    \edge {askers} {questions} ; %
    \edge {questions, answerers} {answers} ; %
    \edge {questions, answers, commenters} {comments} ; %   
  }
\end{figure}

\begin{table}[hbt]
	\centering
	\begin{tabular}{rl}
	\toprule
	\textbf{Symbol} & \textbf{Interpretation}\\ \midrule
	$U_q(t)$ & \# of users who asked questions at time $t$\\
	$U_a(t)$ & \# of users who answered questions at time $t$\\
	$U_c(t)$ & \# of users who made comments at time $t$\\
	$\Delta N_q(t)$ & \# of active questions at time $t$\\
	$\Delta N_a(t)$ & \# of answers to active questions at time $t$\\
	$\Delta N_c^q(t)$ & \# of comments to active questions at time $t$\\
	$\Delta N_c^a(t)$ & \# of comments to active answers at time $t$\\
	$\Delta N_{+v}^q(t)$ & \# of upvotes to active questions at time $t$\\
	$\Delta N_{+v}^a(t)$ & \# of upvotes to active answers at time $t$\\
	$\alpha_{i, j}$ & Coefficient of $i$th input in $j$th induction/reaction\\
	$\beta_{i, j}$ & Coefficient of $i$th output in $j$th induction/reaction\\ \bottomrule
	 \end{tabular}
    \caption{Notations in inductions/reactions}
\end{table}

\begin{abstract}
 In this paper, we analyze a large group of community question answering
 (CQA) websites on Stack Exchange platform through an economic lens by
 modeling them as knowledge markets. While many of these websites
 continually grow over time, we demonstrate that they can, and often will,
 fail at scale. Metrics of CQA health such as the ratio of answers to
 questions, the percentage of answered questions, and the percentage of
 questions with an accepted answer frequently decline as a function of
 system size (no. of participants) on these platforms. 
 Understanding why this phenomenon
 occurs is a necessary step towards preventing the failures at scale. To
 this end, we explore a set of interpretable economic production models to
 capture content generation dynamics in knowledge markets by fitting each
 to the Stack Exchange data. The best performing of these,
 well-known in economic literature as Cobb-Douglas equation,
 provides an intuitive explanation for content generation in the knowledge
 markets. Specifically, it shows that \begin{enumerate*}
  \item factors of content generation such as user participation and content dependency have
   \emph{constant elasticity}---a percentage increase in any of the inputs
   leads to a constant percentage increase in the output,
  \item in many markets, factors exhibit \emph{diminishing returns}---the
   incremental, marginal output decreases as the input is incrementally
   increased, and
  \item markets vary according to their \emph{returns to scale}---the
   increase in output resulting from a proportionate increase in all
   inputs, and finally
  \item many markets exhibit \emph{diseconomies of scale}---measures of
   health decrease as a function of overall
   system size
 \end{enumerate*}.
\end{abstract}

We investigate why knowledge markets such as Stack Overflow can fail at scale. We determine failure based on empirical metrics such as the ratio of answers to questions, percentage of answered questions (questions with at least one answer), and percentage of question with accepted answer (asker marked an answer as accepted). This investigation is important for three reasons. First, despite the large user base, knowledge markets are declining at scale, as evidenced by the empirical metrics. Second, the early work on market failures concentrate on Stack Overflow, and it is unclear if the findings generalize for all knowledge markets. Third, there is no systematic way to understand failures in knowledge markets, and predict these failures in advance. In this paper, we make three contributions. First, we introduce economic production models to capture content generation dynamics in knowledge markets. Second, we empirically study market failure, and based on our model uncover a systematic way to understand market failure. Third, we perform a large-scale analysis of knowledge markets in Stack Exchange—demonstrating the effectiveness of our methods.

\begin{abstract}
 In this paper, we analyze a large group of community question answering
 (CQA) networks on the StackExchange platform through an economic lens by
 modeling them as information markets.  While many of these networks
 continually to grow over time, we demonstrate that they can, and often will,
 fail at scale. Metrics of network success such as the ratio of answers to
 questions, the percentage of answered questions, and the percentage of
 questions with an accepted answer frequently decline as a function of
 network size on these CQA platforms. Understanding why this phenomenon
 occurs is a necessary step towards preventing these failures at scale. To
 this end, we explore a set of interpretable economic production models to
 capture content generation dynamics in knowledge markets by fitting each
 to a collection of StackExchange networks. The best performing of these,
 well-known in economic literature as a form of a Cobb-Douglas equation,
 provides an intuitive explanation for content generation in the knowledge
 markets we explore. Specifically, it shows that \begin{enumerate*}
  \item factors such as user participation and content dependency have
   \emph{constant elasticity}---a percentage increase in any of the inputs
   leads to a constant percentage increase in the output,
  \item in many networks, factors exhibit \emph{diminishing returns}---the
   incremental, marginal output decreases as the input is incrementally
   increased, and
  \item networks vary according to their \emph{returns to scale}---the
   increase in output resulting from a proportionate increase in all
   inputs, and finally
  \item many networks exhibit \emph{diseconomies of scale}---measures of
   network health and exchangeability  decrease as a function of overall
   network size
 \end{enumerate*}.
\end{abstract}


