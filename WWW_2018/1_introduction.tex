%\section{Introduction}
%\textcolor{blue}{To be written.}
%Content generation and subsequent consumption is a key feature of the social web experience. Every second, on average, several thousand tweets are tweeted on Twitter, which corresponds to several hundred million tweets per day and several hundred billion tweets per year. The content generation stats are in similar scale for other massive social media platforms such as Facebook and YouTube. We continue to witness the Web 2.0 era in the light of these user-generated content (UGC) platforms, where content generation is instrumental in preserving sustainability of the platforms-- serving \lq nowness\rq\ in the social web experience. 

%While content generation is at the heart of sustainable UGC platforms, our understanding of what factors are effective at encouraging content generation and associated growth is yet limited. To date, growth of UGC platforms has been studied mainly from the perspective of population dynamics, with researchers emphasizing interacting population based models to explain user growth in these platforms. These models, however, only consider the interaction among users, and ignores the interaction among users and content-- consequently, can not explain content growth. There is a growing body of empirical studies that analyze patterns of user content generation-- primarily temporal patterns-- revealing useful insights. However, even with these insights, there is a gap in realizing the factors of content growth and corresponding dependency. What is missing from the picture are models of content growth that clearly establish the relationship between content growth and associated factors. Developing such models is important for understanding content growth dynamics, predicting future growth, and identifying platforms that will sustain for a long period of time.

\section{Introduction}
%\textcolor{blue}{To be written.}

% What is the problem?

In this paper, we analyze a large group of community question answering (CQA) websites on Stack Exchange network through the Economic lens of a market. Framing Stack Exchange websites as knowledge markets has intuitive appeal: in a hypothetical knowledge market, if no one wants to answer questions, but only ask, or conversely, there are individuals who want to only answer but not ask questions, the ``market'' will collapse. What then, is the relationship among actions (say between questions and answers) in such knowledge markets, for us to deem it healthy? Are larger markets with more participants healthier since there will be more people to ask and answer questions?

%In this paper, we analyze a large network of community question answering websites called Stack Exchanges where individuals exchange knowledge through the Economic lens of a market. 

% why is it important?

Studying CQA websites through an Economic lens allows site operators to reason about whether they should grow the user base. Since most of the popular CQA websites (e.g. Quora, Stack Exchange) do not charge participants, but instead depend on site advertisements for revenue, there is a natural temptation for operators of these websites to grow the user base, so that there is increase in revenue. As we show in this paper, for most Stack Exchange websites, growth in the user base is counter-productive in the sense that they turn unhealthy---specifically, more questions remain unanswered.

Explaining the macroscopic behavior of knowledge markets is important, yet challenging. One can regress some variable of interest (say number of questions) on variables including size, time spent in the website among others. However, explaining why the regression curve looks like the way it does is hard. As we show in this work, using an Economic lens of a market allows us to model dependencies between number of participants, and the amount of content, and to predict the production of content.

% what did you do?

Our main contribution is to model CQA websites as knowledge markets, and to provide insight on the relationship between size and health of these markets. To this end, we develop models to capture content generation dynamics in knowledge markets. We analyze a set of basis functions (the function form of how an input contributes to output) and interaction mechanisms (how the inputs interact with each other), and identify the optimal power basis function and the essential interaction form using a prediction task on the outputs (questions, answers and comments). This form, is the well-known Cobb-Douglas form that connects labor with output. Using the best model fits for each Stack Exchange, we show that the Cobb-Douglas model predicts the production of content with high accuracy and low variance. Furthermore, we can use the Cobb-Douglas parameters of mature (more than 36 months old) markets as a Bayesian prior to improve the prediction for young markets (6-12 months).

%A model for knowledge markets should have the following properties: macro-scale, explanatory, predictive, minimalistic, comprehensive. In Economics, \emph{production} is defined as the process by which human labor is applied, usually with the help of tools and other forms of capital, to produce useful goods or services---the \emph{output}~\cite{stanford2008economics}. We view the participants of a knowledge market to function as labor to generate content such as questions and answers.

The Cobb-Douglas function provides intuitive explanation for content generation in Stack Exchange markets. It demonstrates that in Stack Exchange markets--- \begin{enumerate*}
  \item factors such as user participation and content dependency have \emph{constant elasticity}---percentage increase in any of these inputs will have constant percentage increase in output;
  \item in many markets, factors exhibit \emph{diminishing returns}---decrease in the marginal (incremental) output (e.g., answer production) as an input (e.g. number of people who answer) is incrementally increased, keeping the other inputs constant;
  \item markets vary according to their \emph{returns to scale}---the increase in output resulting from a proportionate increase in all inputs; and
  \item many markets exhibit \emph{diseconomies of scale}---measures of
   health decrease as a function of overall
   system size
 \end{enumerate*}.
  
%If an information market has high returns to scale, then greater efficiency is obtained as the market moves from small to large-scale operation. For example, in Stack Exchange \texttt{academia}, for answer generation, the returns to scale is $0.18+0.65=0.83<1$. The market becomes less efficient as answer generation is expanded, requiring more questions and answerers to increase the number of answers by same amount.

There are two reasons why we see diminishing returns in the Stack Exchange markets. First, the total activity of participants for any Stack Exchange, unsurprisingly follows a power-law pattern. What is interesting is that the power law exponent falls with increase in size for most Stack Exchanges, implying that new users do not participate in the same manner as earlier users. Second, we can identify a stable core of users, who continue to actively participate for long periods of time, contributing to the network health.

Finally, we show diseconomies of scale through experiments on size, analysis of health metrics and exchangeability. For most Stack Exchanges, we see that as size grows, the ratio of answers to questions falls below 1.0, the critical points when some questions go unanswered. Furthermore, using health metrics of number of questions with an accepted answer, and number of questions with at least one answer, we observe that most Stack Exchanges decline in health with increase in size. Finally, we compare the top contributors with the bottom contributors to see if they are ``exchangeable.'' Most Stack Exchanges are not exchangeable in the sense the contributions of the top and the bottom contributors are qualitatively different and differ in absolute terms. These experiments on diseconomies of scale are consistent with the insight from Cobb-Douglas model of production that predicts diminishing returns.

The work is organized as follows. Section 2 presents the related work. Section 3 presents the desired properties of our model. Section 4 presents our proposed models. Section 5 presents the specifics of our dataset. Section 6 presents the empirical results on model fitting. Section 7 explains our model. Section 8 presents the diseconomies of scale. Section 9 presents the implications and limitations of our research. Finally, Section 10 presents our conclusions.

%We organize the rest of this paper as follows.