\section{Introduction}
\textcolor{blue}{To be written.}

% What is the problem?

In this paper, we analyze a large group online social networks called StackExchanges where individuals exchange knowledge through the Economic lens of a market. Framing StackExchange Q\&A networks as information markets has intuitive appeal: in a hypothetical information market, if no one wants to answer questions, but only ask, or conversely, there are individuals who want to only answer but not ask questions, the ``market'' will collapse. What then, is the relationship among actions (say between questions, answers and comments) in such information markets, for us to deem it healthy? Are larger markets with more participants healthier since there will be more people to ask and answer questions?

% why is it important?

Studying knowledge exchange networks through an en Economic lens allows network operators to reason about whether they should grow the network. Since most of the popular social networks (e.g. Facebook, StackExchange) do not charge participants, but instead depend on site advertisements for revenue, there is a natural temptation for operators of these networks to grow the social network, so that there is increase in revenue. As we show in this paper, for most StackExchange networks, growth in the user base is counter-productive in the sense that they turn unhealthy---specifically, more questions remain unanswered.

Explaining the macroscopic behavior of information markets is hard. One can regress some variable of interest (say number of questions) on variables including size, time spent in the network among others. However, explaining why the regression curve looks like the way it does is hard. As we show in this work, using an Economic lens of a market allows us to model dependencies between number of participants, and the amount of content, and to predict the production of content.

% what did you do?

Our main contribution is to model knowledge-exchange networks as information markets, and to provide insight on the relationship between size and health of these markets. A model for content dynamics should have the following properties: macro-scale, explanatory, predictive, minimalistic, comprehensive. In Economics, \emph{production} is defined as the process by which human labor is applied, usually with the help of tools and other forms of capital, to produce useful goods or services---the \emph{output}~\cite{stanford2008economics}. We view the participants of a knowledge market to function as labor to generate content such as questions and answers.

We analyzed a set of basis functions (the function form of how an input contributes to output) and interaction mechanisms (how the inputs interact with each other) and identified the optimal power basis function and the essential interaction form using a prediction task on the outputs (questions, answers and comments). This form, is the well-known Cobb-Douglas form that connects labor with output. Using the best model fits for each stackexchange, we show that the Cobb-Douglas model predicts the production of content with low variance. Furthermore, we can use the Cobb-Douglas parameters of mature (more than 36 months old) markets as a Bayesian prior to improve the prediction for young markets (6-12 months).

The Cobb-Douglas function provides intuitive explanation for content generation in Stack Exchange markets. It demonstrates that in Stack Exchange markets, factors such as user participation and content dependency have \emph{constant elasticity}---percentage increase in any of these inputs will have constant percentage increase in output. Factors exhibit \emph{diminishing returns}---decrease in the marginal (incremental) output content productio as an input (e.g. number of people who answer) is incrementally increased, keeping the other inputs constant. Finally, the Cobb-Douglas model allows us to analyze a market's scale efficiency, as manifested by their \emph{returns to scale}---the increase in output resulting from a proportionate increase in all inputs. If an information market has high returns to scale, then greater efficiency is obtained as the market moves from small to large-scale operation. For example, in StackExchange \texttt{academia}, for answer generation, the returns to scale is $0.18+0.65=0.83<1$. The market becomes less efficient as answer generation is expanded, requiring more questions and answerers to increase the number of answers by same amount.

There are two reasons why we see diminishing returns in the Stack Exchange information market. First, the total activity of participants for any StackExchange, unsurprisingly follows a power-law pattern. What is interesting is that the power law exponent falls with increase in size for most StackExchanges, implying that new users do not participate in the same manner as earlier users. Second, we can identify a stable core of users, who continue to actively participate for long periods of time, contributing to the network health.

Finally, we show diseconomies of scale through experiments on size, analysis of health metrics and exchangeability. For most stackexchanges, we see that as size grows, the ratio of answers to questions falls below 1.0, the critical points when some questions go unanswered. Furthermore, using health metrics of number of questions with an accepted answer, and number of questions with at least one answer, we observe that most StackExchanges decline in health with increase in size. Finally, we compare the top contributors with the bottom contributors to see if they are ``exchangeable.'' Most StackExchanges are not exchangeable in the sense the contributions of the top and the bottom contributors are qualitatively different and differ in absolute terms. These experiments on diseconomies of scale are consistent with the insight from Cobb-Douglas model of production that predicts diminishing returns.

We organize the rest of this paper as follows.




% TOC
