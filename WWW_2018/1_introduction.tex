\section{Introduction}
\textcolor{blue}{To be written.}
Content generation and subsequent consumption is a key feature of the social web experience. Every second, on average, several thousand tweets are tweeted on Twitter, which corresponds to several hundred million tweets per day and several hundred billion tweets per year. The content generation stats are in similar scale for other massive social media platforms such as Facebook and YouTube. We continue to witness the Web 2.0 era in the light of these user-generated content (UGC) platforms, where content generation is instrumental in preserving sustainability of the platforms-- serving \lq nowness\rq\ in the social web experience. 

While content generation is at the heart of sustainable UGC platforms, our understanding of what factors are effective at encouraging content generation and associated growth is yet limited. To date, growth of UGC platforms has been studied mainly from the perspective of population dynamics, with researchers emphasizing interacting population based models to explain user growth in these platforms. These models, however, only consider the interaction among users, and ignores the interaction among users and content-- consequently, can not explain content growth. There is a growing body of empirical studies that analyze patterns of user content generation-- primarily temporal patterns-- revealing useful insights. However, even with these insights, there is a gap in realizing the factors of content growth and corresponding dependency. What is missing from the picture are models of content growth that clearly establish the relationship between content growth and associated factors. Developing such models is important for understanding content growth dynamics, predicting future growth, and identifying platforms that will sustain for a long period of time.