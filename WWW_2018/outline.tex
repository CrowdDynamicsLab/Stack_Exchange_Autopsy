\documentclass[sigconf]{acmart}

\usepackage{booktabs} % For formal tables
% My packages
\usepackage{tikz}
\usetikzlibrary{bayesnet}
\usepackage[normalem]{ulem}
\usepackage{wrapfig}
\usepackage{xcolor}
\usepackage{array}
\newcommand{\hs}[1]{\textcolor{red}{Hari: #1}}

% Copyright
%\setcopyright{none}
%\setcopyright{acmcopyright}
%\setcopyright{acmlicensed}
%\setcopyright{rightsretained}
%\setcopyright{usgov}
%\setcopyright{usgovmixed}
%\setcopyright{cagov}
%\setcopyright{cagovmixed}


% DOI
%\acmDOI{10.475/123_4}

% ISBN
%\acmISBN{123-4567-24-567/08/06}

%Conference
\iffalse
\acmConference[WOODSTOCK'97]{ACM Woodstock conference}{July 1997}{El
  Paso, Texas USA}
\acmYear{1997}
\fi

%\copyrightyear{2016}

%\acmPrice{15.00}


\begin{document}
\title[Too Big to Succeed]{Too Big to Succeed: Understanding Successes and Failures at Scale in Knowledge Markets}

\iffalse
\titlenote{Produces the permission block, and
  copyright information}
\subtitle{Extended Abstract}
\subtitlenote{The full version of the author's guide is available as
  \texttt{acmart.pdf} document}
\fi

\author{Anonymous Author}
\affiliation{%
  \institution{Anonymous Institution}
  \city{City, Country}
}
\email{e-mail}

\iffalse
\author{Ben Trovato}
\authornote{Dr.~Trovato insisted his name be first.}
\orcid{1234-5678-9012}
\affiliation{%
  \institution{Institute for Clarity in Documentation}
  \streetaddress{P.O. Box 1212}
  \city{Dublin}
  \state{Ohio}
  \postcode{43017-6221}
}
\email{trovato@corporation.com}

\author{G.K.M. Tobin}
\authornote{The secretary disavows any knowledge of this author's actions.}
\affiliation{%
  \institution{Institute for Clarity in Documentation}
  \streetaddress{P.O. Box 1212}
  \city{Dublin}
  \state{Ohio}
  \postcode{43017-6221}
}
\email{webmaster@marysville-ohio.com}

\author{Lars Th{\o}rv{\"a}ld}
\authornote{This author is the
  one who did all the really hard work.}
\affiliation{%
  \institution{The Th{\o}rv{\"a}ld Group}
  \streetaddress{1 Th{\o}rv{\"a}ld Circle}
  \city{Hekla}
  \country{Iceland}}
\email{larst@affiliation.org}

\author{Lawrence P. Leipuner}
\affiliation{
  \institution{Brookhaven Laboratories}
  \streetaddress{P.O. Box 5000}}
\email{lleipuner@researchlabs.org}

\author{Sean Fogarty}
\affiliation{%
  \institution{NASA Ames Research Center}
  \city{Moffett Field}
  \state{California}
  \postcode{94035}}
\email{fogartys@amesres.org}

\author{Charles Palmer}
\affiliation{%
  \institution{Palmer Research Laboratories}
  \streetaddress{8600 Datapoint Drive}
  \city{San Antonio}
  \state{Texas}
  \postcode{78229}}
\email{cpalmer@prl.com}

\author{John Smith}
\affiliation{\institution{The Th{\o}rv{\"a}ld Group}}
\email{jsmith@affiliation.org}

\author{Julius P.~Kumquat}
\affiliation{\institution{The Kumquat Consortium}}
\email{jpkumquat@consortium.net}
\fi

% The default list of authors is too long for headers}
%\renewcommand{\shortauthors}{B. Trovato et al.}

\iffalse
\begin{abstract}
This paper provides a sample of a \LaTeX\ document which conforms,
somewhat loosely, to the formatting guidelines for
ACM SIG Proceedings.\footnote{This is an abstract footnote}
\end{abstract}

%
% The code below should be generated by the tool at
% http://dl.acm.org/ccs.cfm
% Please copy and paste the code instead of the example below.
%
\begin{CCSXML}
<ccs2012>
 <concept>
  <concept_id>10010520.10010553.10010562</concept_id>
  <concept_desc>Computer systems organization~Embedded systems</concept_desc>
  <concept_significance>500</concept_significance>
 </concept>
 <concept>
  <concept_id>10010520.10010575.10010755</concept_id>
  <concept_desc>Computer systems organization~Redundancy</concept_desc>
  <concept_significance>300</concept_significance>
 </concept>
 <concept>
  <concept_id>10010520.10010553.10010554</concept_id>
  <concept_desc>Computer systems organization~Robotics</concept_desc>
  <concept_significance>100</concept_significance>
 </concept>
 <concept>
  <concept_id>10003033.10003083.10003095</concept_id>
  <concept_desc>Networks~Network reliability</concept_desc>
  <concept_significance>100</concept_significance>
 </concept>
</ccs2012>
\end{CCSXML}

\ccsdesc[500]{Computer systems organization~Embedded systems}
\ccsdesc[300]{Computer systems organization~Redundancy}
\ccsdesc{Computer systems organization~Robotics}
\ccsdesc[100]{Networks~Network reliability}


\keywords{ACM proceedings, \LaTeX, text tagging}
\fi


\maketitle


%\documentclass[sigconf]{acmart}

\usepackage{booktabs} % For formal tables
% My packages
\usepackage{tikz}
\usetikzlibrary{bayesnet}
\usepackage[normalem]{ulem}
\usepackage{wrapfig}
\usepackage{xcolor}
\usepackage{array}
\newcommand{\hs}[1]{\textcolor{red}{Hari: #1}}

% Copyright
%\setcopyright{none}
%\setcopyright{acmcopyright}
%\setcopyright{acmlicensed}
%\setcopyright{rightsretained}
%\setcopyright{usgov}
%\setcopyright{usgovmixed}
%\setcopyright{cagov}
%\setcopyright{cagovmixed}


% DOI
%\acmDOI{10.475/123_4}

% ISBN
%\acmISBN{123-4567-24-567/08/06}

%Conference
\iffalse
\acmConference[WOODSTOCK'97]{ACM Woodstock conference}{July 1997}{El
  Paso, Texas USA}
\acmYear{1997}
\fi

%\copyrightyear{2016}

%\acmPrice{15.00}


\begin{document}
\title[Too Big to Succeed]{Too Big to Succeed: Understanding Successes and Failures at Scale in Knowledge Markets}

\iffalse
\titlenote{Produces the permission block, and
  copyright information}
\subtitle{Extended Abstract}
\subtitlenote{The full version of the author's guide is available as
  \texttt{acmart.pdf} document}
\fi

\author{Anonymous Author}
\affiliation{%
  \institution{Anonymous Institution}
  \city{City, Country}
}
\email{e-mail}

\iffalse
\author{Ben Trovato}
\authornote{Dr.~Trovato insisted his name be first.}
\orcid{1234-5678-9012}
\affiliation{%
  \institution{Institute for Clarity in Documentation}
  \streetaddress{P.O. Box 1212}
  \city{Dublin}
  \state{Ohio}
  \postcode{43017-6221}
}
\email{trovato@corporation.com}

\author{G.K.M. Tobin}
\authornote{The secretary disavows any knowledge of this author's actions.}
\affiliation{%
  \institution{Institute for Clarity in Documentation}
  \streetaddress{P.O. Box 1212}
  \city{Dublin}
  \state{Ohio}
  \postcode{43017-6221}
}
\email{webmaster@marysville-ohio.com}

\author{Lars Th{\o}rv{\"a}ld}
\authornote{This author is the
  one who did all the really hard work.}
\affiliation{%
  \institution{The Th{\o}rv{\"a}ld Group}
  \streetaddress{1 Th{\o}rv{\"a}ld Circle}
  \city{Hekla}
  \country{Iceland}}
\email{larst@affiliation.org}

\author{Lawrence P. Leipuner}
\affiliation{
  \institution{Brookhaven Laboratories}
  \streetaddress{P.O. Box 5000}}
\email{lleipuner@researchlabs.org}

\author{Sean Fogarty}
\affiliation{%
  \institution{NASA Ames Research Center}
  \city{Moffett Field}
  \state{California}
  \postcode{94035}}
\email{fogartys@amesres.org}

\author{Charles Palmer}
\affiliation{%
  \institution{Palmer Research Laboratories}
  \streetaddress{8600 Datapoint Drive}
  \city{San Antonio}
  \state{Texas}
  \postcode{78229}}
\email{cpalmer@prl.com}

\author{John Smith}
\affiliation{\institution{The Th{\o}rv{\"a}ld Group}}
\email{jsmith@affiliation.org}

\author{Julius P.~Kumquat}
\affiliation{\institution{The Kumquat Consortium}}
\email{jpkumquat@consortium.net}
\fi

% The default list of authors is too long for headers}
%\renewcommand{\shortauthors}{B. Trovato et al.}

\iffalse
\begin{abstract}
This paper provides a sample of a \LaTeX\ document which conforms,
somewhat loosely, to the formatting guidelines for
ACM SIG Proceedings.\footnote{This is an abstract footnote}
\end{abstract}

%
% The code below should be generated by the tool at
% http://dl.acm.org/ccs.cfm
% Please copy and paste the code instead of the example below.
%
\begin{CCSXML}
<ccs2012>
 <concept>
  <concept_id>10010520.10010553.10010562</concept_id>
  <concept_desc>Computer systems organization~Embedded systems</concept_desc>
  <concept_significance>500</concept_significance>
 </concept>
 <concept>
  <concept_id>10010520.10010575.10010755</concept_id>
  <concept_desc>Computer systems organization~Redundancy</concept_desc>
  <concept_significance>300</concept_significance>
 </concept>
 <concept>
  <concept_id>10010520.10010553.10010554</concept_id>
  <concept_desc>Computer systems organization~Robotics</concept_desc>
  <concept_significance>100</concept_significance>
 </concept>
 <concept>
  <concept_id>10003033.10003083.10003095</concept_id>
  <concept_desc>Networks~Network reliability</concept_desc>
  <concept_significance>100</concept_significance>
 </concept>
</ccs2012>
\end{CCSXML}

\ccsdesc[500]{Computer systems organization~Embedded systems}
\ccsdesc[300]{Computer systems organization~Redundancy}
\ccsdesc{Computer systems organization~Robotics}
\ccsdesc[100]{Networks~Network reliability}


\keywords{ACM proceedings, \LaTeX, text tagging}
\fi


\maketitle


%\documentclass[sigconf]{acmart}

\usepackage{booktabs} % For formal tables
% My packages
\usepackage{tikz}
\usetikzlibrary{bayesnet}
\usepackage[normalem]{ulem}
\usepackage{wrapfig}
\usepackage{xcolor}
\usepackage{array}
\newcommand{\hs}[1]{\textcolor{red}{Hari: #1}}

% Copyright
%\setcopyright{none}
%\setcopyright{acmcopyright}
%\setcopyright{acmlicensed}
%\setcopyright{rightsretained}
%\setcopyright{usgov}
%\setcopyright{usgovmixed}
%\setcopyright{cagov}
%\setcopyright{cagovmixed}


% DOI
%\acmDOI{10.475/123_4}

% ISBN
%\acmISBN{123-4567-24-567/08/06}

%Conference
\iffalse
\acmConference[WOODSTOCK'97]{ACM Woodstock conference}{July 1997}{El
  Paso, Texas USA}
\acmYear{1997}
\fi

%\copyrightyear{2016}

%\acmPrice{15.00}


\begin{document}
\title[Too Big to Succeed]{Too Big to Succeed: Understanding Successes and Failures at Scale in Knowledge Markets}

\iffalse
\titlenote{Produces the permission block, and
  copyright information}
\subtitle{Extended Abstract}
\subtitlenote{The full version of the author's guide is available as
  \texttt{acmart.pdf} document}
\fi

\author{Anonymous Author}
\affiliation{%
  \institution{Anonymous Institution}
  \city{City, Country}
}
\email{e-mail}

\iffalse
\author{Ben Trovato}
\authornote{Dr.~Trovato insisted his name be first.}
\orcid{1234-5678-9012}
\affiliation{%
  \institution{Institute for Clarity in Documentation}
  \streetaddress{P.O. Box 1212}
  \city{Dublin}
  \state{Ohio}
  \postcode{43017-6221}
}
\email{trovato@corporation.com}

\author{G.K.M. Tobin}
\authornote{The secretary disavows any knowledge of this author's actions.}
\affiliation{%
  \institution{Institute for Clarity in Documentation}
  \streetaddress{P.O. Box 1212}
  \city{Dublin}
  \state{Ohio}
  \postcode{43017-6221}
}
\email{webmaster@marysville-ohio.com}

\author{Lars Th{\o}rv{\"a}ld}
\authornote{This author is the
  one who did all the really hard work.}
\affiliation{%
  \institution{The Th{\o}rv{\"a}ld Group}
  \streetaddress{1 Th{\o}rv{\"a}ld Circle}
  \city{Hekla}
  \country{Iceland}}
\email{larst@affiliation.org}

\author{Lawrence P. Leipuner}
\affiliation{
  \institution{Brookhaven Laboratories}
  \streetaddress{P.O. Box 5000}}
\email{lleipuner@researchlabs.org}

\author{Sean Fogarty}
\affiliation{%
  \institution{NASA Ames Research Center}
  \city{Moffett Field}
  \state{California}
  \postcode{94035}}
\email{fogartys@amesres.org}

\author{Charles Palmer}
\affiliation{%
  \institution{Palmer Research Laboratories}
  \streetaddress{8600 Datapoint Drive}
  \city{San Antonio}
  \state{Texas}
  \postcode{78229}}
\email{cpalmer@prl.com}

\author{John Smith}
\affiliation{\institution{The Th{\o}rv{\"a}ld Group}}
\email{jsmith@affiliation.org}

\author{Julius P.~Kumquat}
\affiliation{\institution{The Kumquat Consortium}}
\email{jpkumquat@consortium.net}
\fi

% The default list of authors is too long for headers}
%\renewcommand{\shortauthors}{B. Trovato et al.}

\iffalse
\begin{abstract}
This paper provides a sample of a \LaTeX\ document which conforms,
somewhat loosely, to the formatting guidelines for
ACM SIG Proceedings.\footnote{This is an abstract footnote}
\end{abstract}

%
% The code below should be generated by the tool at
% http://dl.acm.org/ccs.cfm
% Please copy and paste the code instead of the example below.
%
\begin{CCSXML}
<ccs2012>
 <concept>
  <concept_id>10010520.10010553.10010562</concept_id>
  <concept_desc>Computer systems organization~Embedded systems</concept_desc>
  <concept_significance>500</concept_significance>
 </concept>
 <concept>
  <concept_id>10010520.10010575.10010755</concept_id>
  <concept_desc>Computer systems organization~Redundancy</concept_desc>
  <concept_significance>300</concept_significance>
 </concept>
 <concept>
  <concept_id>10010520.10010553.10010554</concept_id>
  <concept_desc>Computer systems organization~Robotics</concept_desc>
  <concept_significance>100</concept_significance>
 </concept>
 <concept>
  <concept_id>10003033.10003083.10003095</concept_id>
  <concept_desc>Networks~Network reliability</concept_desc>
  <concept_significance>100</concept_significance>
 </concept>
</ccs2012>
\end{CCSXML}

\ccsdesc[500]{Computer systems organization~Embedded systems}
\ccsdesc[300]{Computer systems organization~Redundancy}
\ccsdesc{Computer systems organization~Robotics}
\ccsdesc[100]{Networks~Network reliability}


\keywords{ACM proceedings, \LaTeX, text tagging}
\fi


\maketitle


%\documentclass[sigconf]{acmart}

\usepackage{booktabs} % For formal tables
% My packages
\usepackage{tikz}
\usetikzlibrary{bayesnet}
\usepackage[normalem]{ulem}
\usepackage{wrapfig}
\usepackage{xcolor}
\usepackage{array}
\newcommand{\hs}[1]{\textcolor{red}{Hari: #1}}

% Copyright
%\setcopyright{none}
%\setcopyright{acmcopyright}
%\setcopyright{acmlicensed}
%\setcopyright{rightsretained}
%\setcopyright{usgov}
%\setcopyright{usgovmixed}
%\setcopyright{cagov}
%\setcopyright{cagovmixed}


% DOI
%\acmDOI{10.475/123_4}

% ISBN
%\acmISBN{123-4567-24-567/08/06}

%Conference
\iffalse
\acmConference[WOODSTOCK'97]{ACM Woodstock conference}{July 1997}{El
  Paso, Texas USA}
\acmYear{1997}
\fi

%\copyrightyear{2016}

%\acmPrice{15.00}


\begin{document}
\title[Too Big to Succeed]{Too Big to Succeed: Understanding Successes and Failures at Scale in Knowledge Markets}

\iffalse
\titlenote{Produces the permission block, and
  copyright information}
\subtitle{Extended Abstract}
\subtitlenote{The full version of the author's guide is available as
  \texttt{acmart.pdf} document}
\fi

\author{Anonymous Author}
\affiliation{%
  \institution{Anonymous Institution}
  \city{City, Country}
}
\email{e-mail}

\iffalse
\author{Ben Trovato}
\authornote{Dr.~Trovato insisted his name be first.}
\orcid{1234-5678-9012}
\affiliation{%
  \institution{Institute for Clarity in Documentation}
  \streetaddress{P.O. Box 1212}
  \city{Dublin}
  \state{Ohio}
  \postcode{43017-6221}
}
\email{trovato@corporation.com}

\author{G.K.M. Tobin}
\authornote{The secretary disavows any knowledge of this author's actions.}
\affiliation{%
  \institution{Institute for Clarity in Documentation}
  \streetaddress{P.O. Box 1212}
  \city{Dublin}
  \state{Ohio}
  \postcode{43017-6221}
}
\email{webmaster@marysville-ohio.com}

\author{Lars Th{\o}rv{\"a}ld}
\authornote{This author is the
  one who did all the really hard work.}
\affiliation{%
  \institution{The Th{\o}rv{\"a}ld Group}
  \streetaddress{1 Th{\o}rv{\"a}ld Circle}
  \city{Hekla}
  \country{Iceland}}
\email{larst@affiliation.org}

\author{Lawrence P. Leipuner}
\affiliation{
  \institution{Brookhaven Laboratories}
  \streetaddress{P.O. Box 5000}}
\email{lleipuner@researchlabs.org}

\author{Sean Fogarty}
\affiliation{%
  \institution{NASA Ames Research Center}
  \city{Moffett Field}
  \state{California}
  \postcode{94035}}
\email{fogartys@amesres.org}

\author{Charles Palmer}
\affiliation{%
  \institution{Palmer Research Laboratories}
  \streetaddress{8600 Datapoint Drive}
  \city{San Antonio}
  \state{Texas}
  \postcode{78229}}
\email{cpalmer@prl.com}

\author{John Smith}
\affiliation{\institution{The Th{\o}rv{\"a}ld Group}}
\email{jsmith@affiliation.org}

\author{Julius P.~Kumquat}
\affiliation{\institution{The Kumquat Consortium}}
\email{jpkumquat@consortium.net}
\fi

% The default list of authors is too long for headers}
%\renewcommand{\shortauthors}{B. Trovato et al.}

\iffalse
\begin{abstract}
This paper provides a sample of a \LaTeX\ document which conforms,
somewhat loosely, to the formatting guidelines for
ACM SIG Proceedings.\footnote{This is an abstract footnote}
\end{abstract}

%
% The code below should be generated by the tool at
% http://dl.acm.org/ccs.cfm
% Please copy and paste the code instead of the example below.
%
\begin{CCSXML}
<ccs2012>
 <concept>
  <concept_id>10010520.10010553.10010562</concept_id>
  <concept_desc>Computer systems organization~Embedded systems</concept_desc>
  <concept_significance>500</concept_significance>
 </concept>
 <concept>
  <concept_id>10010520.10010575.10010755</concept_id>
  <concept_desc>Computer systems organization~Redundancy</concept_desc>
  <concept_significance>300</concept_significance>
 </concept>
 <concept>
  <concept_id>10010520.10010553.10010554</concept_id>
  <concept_desc>Computer systems organization~Robotics</concept_desc>
  <concept_significance>100</concept_significance>
 </concept>
 <concept>
  <concept_id>10003033.10003083.10003095</concept_id>
  <concept_desc>Networks~Network reliability</concept_desc>
  <concept_significance>100</concept_significance>
 </concept>
</ccs2012>
\end{CCSXML}

\ccsdesc[500]{Computer systems organization~Embedded systems}
\ccsdesc[300]{Computer systems organization~Redundancy}
\ccsdesc{Computer systems organization~Robotics}
\ccsdesc[100]{Networks~Network reliability}


\keywords{ACM proceedings, \LaTeX, text tagging}
\fi


\maketitle


%\input{outline}


\section{Introduction}

\section{Related Work}

\section{Problem Formulation}
In this section we describe the desiderata for designing a model to understand the successes/failures of knowledge markets.

\section{Modeling Knowledge Markets}
In this section we introduce production models to capture content generation dynamics in real-world knowledge markets. We first draw an analogy between economic production and content generation (Section 4.1), and then report the content generation factors in knowledge markets (Section 4.2). Next, we concentrate on the knowledge markets in Stack Exchange networks---presenting production models to capture content generation dynamics for different content types (Section 4.3).
\subsection{Production Analogy}
We conceptualize content generation in knowledge markets as economic production.
\subsection{Factors of Content Generation}
We recognize the key factors of content generation in knowledge markets.
\subsection{Modeling Markets in Stack Exchange}
Now, we concentrate on modeling the knowledge markets in Stack Exchange, where each market primarily generates three types of contents: question, answer, and comment.

\section{Dataset}
We collected the latest release (September, 2017) of Stack Exchange dataset.

\section{Evaluating Our Proposed Models}
In this section we examine our proposed models from three different perspectives: the accuracy of fitting content generation time series observed in our dataset (Section 6.1), the performance of predicting content volume in long and short run (Section 6.2), and the perplexity of characterizing content generation dynamics at early stage (Section 6.3).
\subsection{Model Fitting}
We fit each variant of production model for each content type to the observed time series in each Stack Exchange website.
\subsection{Forecasting Content Generation}
We apply the best-fit production models to predict content volume in long and short run.
\subsection{Parameter Estimation for New Websites}
We use parameters learnt from old Stack Exchange websites as priors for new Stack Exchange websites.

\section{Characterizing Knowledge Markets}
In this section we characterize the knowledge markets in Stack Exchange---explaining the best-fit models and their foundations (Section 7.1), revealing two key distributions that control the markets (Section 7.2), and uncovering the stable core that maintains market equilibrium (Section 7.3).
\subsection{Model Interpretation}
First, we explain the best-fit models found in Section 6.1.
\subsection{Two Key Distributions}
Next, we discuss two key distributions that control content generation in knowledge markets, namely participant activity and subject POV (perspective).
\subsection{Uncovering the Stable Core}
Now, we show the presence of a stable core of users that control the dynamic market equilibrium hypothesized by the Cobb-Douglas function.

\section{Diseconomies of Scale}
In this section we discuss the diseconomies of scale that occur in the knowledge markets.
\subsection{Empirical Observation}
Backed by the diminishing returns, Stack Exchange websites undergo diseconomies of scale---the ratio of answers to questions go down with the increase in number of users.
\subsection{Decline in Health}
As the health of knowledge markets directly depend on content generation, we investigate the effect of scale on a set of health metrics.
\subsection{Decline in Stability}
As the stability of knowledge markets depend on user participation, we investigate the effect of scale on stability.

\section{Discussion}
In this section we discuss the analytical findings of our model (Section 8.1) along with a couple of complementary and alternative models (Section 8.2).
\subsection{Analytical Findings}
We present several analytical findings that have implications for different aspects of knowledge markets.
\subsection{Complementary and Alternative Models}
We consider several alternative models to comprehend the content generation dynamics in knowledge markets.

\section{Conclusion}
\bibliographystyle{ACM-Reference-Format}
\bibliography{sigproc}

\end{document}



\section{Introduction}

\section{Related Work}

\section{Problem Formulation}
In this section we describe the desiderata for designing a model to understand the successes/failures of knowledge markets.

\section{Modeling Knowledge Markets}
In this section we introduce production models to capture content generation dynamics in real-world knowledge markets. We first draw an analogy between economic production and content generation (Section 4.1), and then report the content generation factors in knowledge markets (Section 4.2). Next, we concentrate on the knowledge markets in Stack Exchange networks---presenting production models to capture content generation dynamics for different content types (Section 4.3).
\subsection{Production Analogy}
We conceptualize content generation in knowledge markets as economic production.
\subsection{Factors of Content Generation}
We recognize the key factors of content generation in knowledge markets.
\subsection{Modeling Markets in Stack Exchange}
Now, we concentrate on modeling the knowledge markets in Stack Exchange, where each market primarily generates three types of contents: question, answer, and comment.

\section{Dataset}
We collected the latest release (September, 2017) of Stack Exchange dataset.

\section{Evaluating Our Proposed Models}
In this section we examine our proposed models from three different perspectives: the accuracy of fitting content generation time series observed in our dataset (Section 6.1), the performance of predicting content volume in long and short run (Section 6.2), and the perplexity of characterizing content generation dynamics at early stage (Section 6.3).
\subsection{Model Fitting}
We fit each variant of production model for each content type to the observed time series in each Stack Exchange website.
\subsection{Forecasting Content Generation}
We apply the best-fit production models to predict content volume in long and short run.
\subsection{Parameter Estimation for New Websites}
We use parameters learnt from old Stack Exchange websites as priors for new Stack Exchange websites.

\section{Characterizing Knowledge Markets}
In this section we characterize the knowledge markets in Stack Exchange---explaining the best-fit models and their foundations (Section 7.1), revealing two key distributions that control the markets (Section 7.2), and uncovering the stable core that maintains market equilibrium (Section 7.3).
\subsection{Model Interpretation}
First, we explain the best-fit models found in Section 6.1.
\subsection{Two Key Distributions}
Next, we discuss two key distributions that control content generation in knowledge markets, namely participant activity and subject POV (perspective).
\subsection{Uncovering the Stable Core}
Now, we show the presence of a stable core of users that control the dynamic market equilibrium hypothesized by the Cobb-Douglas function.

\section{Diseconomies of Scale}
In this section we discuss the diseconomies of scale that occur in the knowledge markets.
\subsection{Empirical Observation}
Backed by the diminishing returns, Stack Exchange websites undergo diseconomies of scale---the ratio of answers to questions go down with the increase in number of users.
\subsection{Decline in Health}
As the health of knowledge markets directly depend on content generation, we investigate the effect of scale on a set of health metrics.
\subsection{Decline in Stability}
As the stability of knowledge markets depend on user participation, we investigate the effect of scale on stability.

\section{Discussion}
In this section we discuss the analytical findings of our model (Section 8.1) along with a couple of complementary and alternative models (Section 8.2).
\subsection{Analytical Findings}
We present several analytical findings that have implications for different aspects of knowledge markets.
\subsection{Complementary and Alternative Models}
We consider several alternative models to comprehend the content generation dynamics in knowledge markets.

\section{Conclusion}
\bibliographystyle{ACM-Reference-Format}
\bibliography{sigproc}

\end{document}



\section{Introduction}

\section{Related Work}

\section{Problem Formulation}
In this section we describe the desiderata for designing a model to understand the successes/failures of knowledge markets.

\section{Modeling Knowledge Markets}
In this section we introduce production models to capture content generation dynamics in real-world knowledge markets. We first draw an analogy between economic production and content generation (Section 4.1), and then report the content generation factors in knowledge markets (Section 4.2). Next, we concentrate on the knowledge markets in Stack Exchange networks---presenting production models to capture content generation dynamics for different content types (Section 4.3).
\subsection{Production Analogy}
We conceptualize content generation in knowledge markets as economic production.
\subsection{Factors of Content Generation}
We recognize the key factors of content generation in knowledge markets.
\subsection{Modeling Markets in Stack Exchange}
Now, we concentrate on modeling the knowledge markets in Stack Exchange, where each market primarily generates three types of contents: question, answer, and comment.

\section{Dataset}
We collected the latest release (September, 2017) of Stack Exchange dataset.

\section{Evaluating Our Proposed Models}
In this section we examine our proposed models from three different perspectives: the accuracy of fitting content generation time series observed in our dataset (Section 6.1), the performance of predicting content volume in long and short run (Section 6.2), and the perplexity of characterizing content generation dynamics at early stage (Section 6.3).
\subsection{Model Fitting}
We fit each variant of production model for each content type to the observed time series in each Stack Exchange website.
\subsection{Forecasting Content Generation}
We apply the best-fit production models to predict content volume in long and short run.
\subsection{Parameter Estimation for New Websites}
We use parameters learnt from old Stack Exchange websites as priors for new Stack Exchange websites.

\section{Characterizing Knowledge Markets}
In this section we characterize the knowledge markets in Stack Exchange---explaining the best-fit models and their foundations (Section 7.1), revealing two key distributions that control the markets (Section 7.2), and uncovering the stable core that maintains market equilibrium (Section 7.3).
\subsection{Model Interpretation}
First, we explain the best-fit models found in Section 6.1.
\subsection{Two Key Distributions}
Next, we discuss two key distributions that control content generation in knowledge markets, namely participant activity and subject POV (perspective).
\subsection{Uncovering the Stable Core}
Now, we show the presence of a stable core of users that control the dynamic market equilibrium hypothesized by the Cobb-Douglas function.

\section{Diseconomies of Scale}
In this section we discuss the diseconomies of scale that occur in the knowledge markets.
\subsection{Empirical Observation}
Backed by the diminishing returns, Stack Exchange websites undergo diseconomies of scale---the ratio of answers to questions go down with the increase in number of users.
\subsection{Decline in Health}
As the health of knowledge markets directly depend on content generation, we investigate the effect of scale on a set of health metrics.
\subsection{Decline in Stability}
As the stability of knowledge markets depend on user participation, we investigate the effect of scale on stability.

\section{Discussion}
In this section we discuss the analytical findings of our model (Section 8.1) along with a couple of complementary and alternative models (Section 8.2).
\subsection{Analytical Findings}
We present several analytical findings that have implications for different aspects of knowledge markets.
\subsection{Complementary and Alternative Models}
We consider several alternative models to comprehend the content generation dynamics in knowledge markets.

\section{Conclusion}
\bibliographystyle{ACM-Reference-Format}
\bibliography{sigproc}

\end{document}



\section{Introduction}

\section{Related Work}

\section{Problem Formulation}
In this section we describe the desiderata for designing a model to understand the successes/failures of knowledge markets.

\section{Modeling Knowledge Markets}
In this section we introduce production models to capture content generation dynamics in real-world knowledge markets. We first draw an analogy between economic production and content generation (Section 4.1), and then report the content generation factors in knowledge markets (Section 4.2). Next, we concentrate on the knowledge markets in Stack Exchange networks---presenting production models to capture content generation dynamics for different content types (Section 4.3).
\subsection{Production Analogy}
We conceptualize content generation in knowledge markets as economic production.
\subsection{Factors of Content Generation}
We recognize the key factors of content generation in knowledge markets.
\subsection{Modeling Markets in Stack Exchange}
Now, we concentrate on modeling the knowledge markets in Stack Exchange, where each market primarily generates three types of contents: question, answer, and comment.

\section{Dataset}
We collected the latest release (September, 2017) of Stack Exchange dataset.

\section{Evaluating Our Proposed Models}
In this section we examine our proposed models from three different perspectives: the accuracy of fitting content generation time series observed in our dataset (Section 6.1), the performance of predicting content volume in long and short run (Section 6.2), and the perplexity of characterizing content generation dynamics at early stage (Section 6.3).
\subsection{Model Fitting}
We fit each variant of production model for each content type to the observed time series in each Stack Exchange website.
\subsection{Forecasting Content Generation}
We apply the best-fit production models to predict content volume in long and short run.
\subsection{Parameter Estimation for New Websites}
We use parameters learnt from old Stack Exchange websites as priors for new Stack Exchange websites.

\section{Characterizing Knowledge Markets}
In this section we characterize the knowledge markets in Stack Exchange---explaining the best-fit models and their foundations (Section 7.1), revealing two key distributions that control the markets (Section 7.2), and uncovering the stable core that maintains market equilibrium (Section 7.3).
\subsection{Model Interpretation}
First, we explain the best-fit models found in Section 6.1.
\subsection{Two Key Distributions}
Next, we discuss two key distributions that control content generation in knowledge markets, namely participant activity and subject POV (perspective).
\subsection{Uncovering the Stable Core}
Now, we show the presence of a stable core of users that control the dynamic market equilibrium hypothesized by the Cobb-Douglas function.

\section{Diseconomies of Scale}
In this section we discuss the diseconomies of scale that occur in the knowledge markets.
\subsection{Empirical Observation}
Backed by the diminishing returns, Stack Exchange websites undergo diseconomies of scale---the ratio of answers to questions go down with the increase in number of users.
\subsection{Decline in Health}
As the health of knowledge markets directly depend on content generation, we investigate the effect of scale on a set of health metrics.
\subsection{Decline in Stability}
As the stability of knowledge markets depend on user participation, we investigate the effect of scale on stability.

\section{Discussion}
In this section we discuss the analytical findings of our model (Section 8.1) along with a couple of complementary and alternative models (Section 8.2).
\subsection{Analytical Findings}
We present several analytical findings that have implications for different aspects of knowledge markets.
\subsection{Complementary and Alternative Models}
We consider several alternative models to comprehend the content generation dynamics in knowledge markets.

\section{Conclusion}
\bibliographystyle{ACM-Reference-Format}
\bibliography{sigproc}

\end{document}
