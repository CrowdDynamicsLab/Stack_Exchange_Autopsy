\section{Introduction}

\section{Related Work}

\section{Problem Formulation} 
In this section we describe the desiderata for designing a model to understand the successes/failures of knowledge markets.

\section{Modeling Knowledge Markets}
In this section we introduce production models to capture content generation dynamics in real-world knowledge markets. We first draw an analogy between economic production and content generation (Section 4.1), and then report the content generation factors in knowledge markets (Section 4.2). Next, we concentrate on the knowledge markets in Stack Exchange networks---presenting production models to capture content generation dynamics for different content types (Section 4.3).
\subsection{Production Analogy} 
We conceptualize content generation in knowledge markets as economic production.
\subsection{Factors of Content Generation} 
We recognize the key factors of content generation in knowledge markets.
\subsection{Modeling Markets in Stack Exchange}
Now, we concentrate on modeling the knowledge markets in Stack Exchange, where each market primarily generates three types of contents: question, answer, and comment. 

\section{Dataset} 
We collected the latest release (September, 2017) of Stack Exchange dataset. 

\section{Evaluating Our Proposed Models}
In this section we examine our proposed models from three different perspectives: the accuracy of fitting content generation time series observed in our dataset (Section 6.1), the performance of predicting content volume in long and short run (Section 6.2), and the perplexity of characterizing content generation dynamics at early stage (Section 6.3).
\subsection{Model Fitting}
We fit each variant of production model for each content type to the observed time series in each Stack Exchange website. 
\subsection{Forecasting Content Generation} 
We apply the best-fit production models to predict content volume in long and short run. 
\subsection{Parameter Estimation for New Websites} 
We use parameters learnt from old Stack Exchange websites as priors for new Stack Exchange websites.

\section{Characterizing Knowledge Markets}
In this section we characterize the knowledge markets in Stack Exchange---explaining the best-fit models and their foundations (Section 7.1), revealing two key distributions that control the markets (Section 7.2), and uncovering the stable core that maintains market equilibrium (Section 7.3).
\subsection{Model Interpretation} 
First, we explain the best-fit models found in Section 6.1. 
\subsection{Two Key Distributions} 
Next, we discuss two key distributions that control content generation in knowledge markets, namely participant activity and subject POV (perspective). 
\subsection{Uncovering the Stable Core} 
Now, we show the presence of a stable core of users that control the dynamic market equilibrium hypothesized by the Cobb-Douglas function.

\section{Diseconomies of Scale}
In this section we discuss the diseconomies of scale that occur in the knowledge markets.
\subsection{Empirical Observation}
Backed by the diminishing returns, Stack Exchange websites undergo diseconomies of scale---the ratio of answers to questions go down with the increase in number of users.
\subsection{Decline in Health} 
As the health of knowledge markets directly depend on content generation, we investigate the effect of scale on a set of health metrics.
\subsection{Decline in Stability} 
As the stability of knowledge markets depend on user participation, we investigate the effect of scale on stability.

\section{Discussion}
In this section we discuss the analytical findings of our model (Section 8.1) along with a couple of complementary and alternative models (Section 8.2).
\subsection{Analytical Findings}
We present several analytical findings that have implications for different aspects of knowledge markets.
\subsection{Complementary and Alternative Models}
We consider several alternative models to comprehend the content generation dynamics in knowledge markets. 

\section{Conclusion}



