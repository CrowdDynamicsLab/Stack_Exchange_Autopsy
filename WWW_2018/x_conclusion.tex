\section{Conclusion}

% Chase's version
In this paper, we examined CQA websites on the StackExchange platform through an economic lens by modeling them as knowledge markets. We analyzed a set of basis functions (the functional form of how an input factor, such as the number of available answerers, contributes to the overall output of the network) and interaction mechanisms (how the input  factors interact with each other) to capture the content generation dynamics in knowledge markets. The resulting best-fit model, Cobb-Douglas, predicts the production of content with high accuracy. In addition, we showed that the model provides intuitive explanations for content generation in Stack Exchange markets. Namely, \begin{enumerate*}
    \item factors such as user participation and content dependency have \emph{constant elasticity}, meaning that a percentage increase in any of these inputs will result in a constant percentage increase in output;
    \item input factors exhibit \emph{diminishing returns} in that there is a decrease in the marginal (incremental) output (content production) as an input (e.g. number of people who answer) is incrementally increased while holding the other inputs constant; and
    \item the efficiency of markets varies as manifest by their \emph{returns to scale}---the increase in output resulting from a proportionate increase in all inputs
\end{enumerate*}. Finally, we explore the implications of the Cobb-Douglas economic model by showing the presence of diseconomies of scale in terms of the production of content, several measures of network health, and finally the exchangability of users in the network. We conclude that there is more to a successful CQA network than merely the total number of active users. Blindly increasing network sizes and ignoring these potential diseconomies of scale is unwise if we wish to sustain healthy knowledge markets.

% Himel's version
%In this paper, we examine CQA websites in Stack Exchange through an economic lens by modeling them as knowledge markets. We analyze a set of basis functions (the function form of how an input contributes to output) and interaction mechanisms (how the inputs interact with each other) to capture content generation dynamics in knowledge markets. The resultant best-fit model, Cobb-Douglas, predicts the production of content with high accuracy. In addition, the model provides intuitive explanation for content generation in Stack Exchange markets. Specifically, it shows that factors of content generation such as user participation and content dependency have \emph{constant elasticity}; in many markets, factors exhibit \emph{diminishing returns}; markets vary according to their \emph{returns to scale}; and finally many markets exhibit \emph{diseconomies of scale}. 
